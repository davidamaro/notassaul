\section{Schrödinger equation}
\subsection{Time evolution}
In qm time is a parameter and not an observable/operator.
$H$ labels a state given at a time. This implies that we may
envirage that, if a physical system is represented by
$\vert \alpha\rangle $ at the time $t_0$, at a later time, the state may
differ. The new ``evolved" state from $t_0$ to $t$ can be
denoted by
$$
\vert \alpha,t_0;t\rangle , \quad t \geq t_0
$$
such that
$$
\vert \alpha,t_0;t_0\rangle  = \vert \alpha\rangle 
$$
The transition from $\vert \alpha\rangle $ to $\vert \alpha,t_0;t\rangle $ is given
by the so-called time-evolution operator $\hat{U}(t,t_0)$,
such that
\begin{equation}
  \vert \alpha,t_0;t\rangle  = \hat{U}(t,t_0) \vert \alpha\rangle 
  \label{equ:2.1}
\end{equation}
The time-evolution operator has the following properties:
\begin{itemize}
  \item Unitarity. Suppose that at $t_0$ $\vert \alpha\rangle $ is
    expanded in terms of the (orthonormal) eigenstates $\vert a\rangle $
    of some observable $A$
    \begin{equation}
      \vert \alpha\rangle  = \sum_a \vert a\rangle \langle a\vert \alpha,t_0;t\rangle  = \sum_a
      c_a(t_0)\vert a\rangle 
      \label{equ:2.2}
    \end{equation}
    Thus, at a later time $t \rangle  t_0$ we have
    \begin{equation}
      \vert \alpha,t_0;t\rangle  = \sum_a c_a(t) \vert a\rangle 
      \label{equ:2.3}
    \end{equation}
    where $c_a(t)$ is the prob. amp. of finding $\vert \alpha\rangle $
    at $\vert a\rangle $.
    Although we expect that in general for any particular
    $a$, the prob. amplitudes differ,
    $$-
      c_a(t) \neq c_a(t_0)
    $$
    the sum of all probabilies must be unity at all times,
    i.e.,
    \begin{equation}
      \sum_a \vert c_a(t)\vert ^2 = \sum_a \vert c_a(t_0)\vert ^2 = 1
      \label{equ:2.4}
    \end{equation}
    (as long as the states $\vert \alpha\rangle $ and $\vert a\rangle $ are
    "correctly" normalized).
    This implies that
    \begin{equation}
      \langle \alpha\vert \alpha\rangle  = \sum_{a'} \sum_a c_{a'}^*(t_0)
      \langle a'\vert a\rangle   c_a(t_0) = \sum_a \vert c_a(t_0)\vert ^2 = 1
      \label{equ:2.5}
    \end{equation}
    and consequently
    \begin{equation}
      \langle \alpha,t_0;t\vert \alpha,t_0;t\rangle  = \sum_a \vert c_a(t)\vert ^2 = 1
      \label{equ:2.6}
    \end{equation}
    That is, the state $\vert \alpha\rangle $ remain normalized at all
    times. For terms of \ref{equ:2.1}, this is satisfied as
    long as
    \begin{equation}
      \hat{U^\dagger}(t, t_0) \hat{U}(t,t_0) = 1
      \label{equ:2.7}
    \end{equation}
  \item Time composition:
    \begin{equation}
      \hat{U}(t_2,t_0) = \hat{U}(t_2,t_1) \hat{U}(t_1,t_0)
      \qquad t_0 \langle  t_1 \langle  t_2
      \label{equ:2.8}
    \end{equation}
  \item Infinitesimal form. We can also propose an
    infinitesimal form for $\hat{U}(t,t_0)$, such that
    \begin{equation}
      \vert \alpha,t_0;t_0 + dt\rangle  = \hat{U}(t_0+dt,t_0) \vert \alpha\rangle ,
      \qquad \lim_{dt \to 0} \hat{U}(t_0+dt,t_0) = 1
      \label{equ:2.9}
    \end{equation}
    We can note that this together with the previous
    properties is satisfied if
    \begin{equation}
      \hat{U}(t_0+dt,t_0) = 1- i\hat{\Omega)dt}
      \label{equ:2.10}
    \end{equation}
    when $\hat{\Omega}$ is a Hermitian operator.
    This can be shown as follows. Notice that
    \begin{align*}
      \hat{U^\dagger} (t_0+dt,t_0) \hat{U}(t_0+dt,t_0) &= 
      (1+ i\hat{\Omega)dt})(1- i\hat{\Omega)dt})\\
      &=
      1+i(\hat{\Omega^\dagger}-\hat{\Omega})dt + O(dt^2)
    \end{align*}
    which equals the identity as long as $\hat{\Omega} =
    \hat{\Omega^\dagger}$ (Hermitian) and $dt^2$ is
    negligilble. Furthermore, the time composition is also
    direct
    \begin{align*}
      \hat{U}(t_0+dt_1+dt_2,t_0+dt_1)\hat{U}(t_0+dt_1,t_0)
      &= (1-i\hat{\Omega}dt_2)(1-i\hat{\Omega}dt_1) \\
      &= 1-i\hat{\Omega}(dt_2+dt_1) + O(dt_1dt_2)\\
      &= \hat{U}(t_0+dt_1+dt_2,t_0)
    \end{align*}
    Borrwing from classical mechanics the idea that the
    Hamiltonian $H$ is the generator of time evolution, we
    can asser that
    \begin{equation}
      \hat{\Omega} = \frac{\hat{H}}{\hbar}
      \label{equ:2.11}
    \end{equation}
    which has the righ dimensions of frequency. Therefore,
    we obtain
    \begin{equation}
      \hat{U}(t_0+dt,t_0) = 1 -\frac{i}{\hbar}\hat{H}dt
      \label{equ:2.12}
    \end{equation}
    for (\ref{equ:2.11}) we have used $\hbar$ to get the
    right dimensions, recalling the Planck radiation $E =
    \hbar \omega$.
\end{itemize}
These properties lead to and interesting differential
quation. Using the time decomposition property, we see that
\begin{equation}
  \hat{U}(t+dt,t_0) = \hat{U}(t+dt,t) \hat{U}(t,t_0) =
  (1-\frac{i}{\hbar}\hat{H}dt)\hat{U}(t,t_0)
  \label{equ:2.13}
\end{equation}
Note now that
\begin{equation}
  \hat{U}(t+dt,t_0)-\hat{U}(t,t_0) = - \frac{i}{\hbar}
  \hat{H}dt \hat{U}(t,t_0)
  \label{equ:2.14}
\end{equation}
which implies
$$
i\hbar \frac{\hat{U}(t+dt,t_0)-\hat{U}(t,t_0)}{dt} =
\hat{H}\hat{U}(t,t_0)
$$
that can be written in the differential form
\begin{equation}
  i\hbar \frac{\partial}{\partial t}\hat{U}(t_0,t) = \hat{H}
  \hat{U}(t,t_0)
  \label{equ:2.15}
\end{equation}
This is the Schrödinger equation for the time-evolution
operator and is the most fundamental equation of quantum
mechanics.
Multiplying (\ref{equ:2.15}) by the ket $\vert \alpha\rangle  =
%todo no entiendo que es lo que dice el antes del we find
\vert \alpha,t_0;t_1\rangle $, we find
$$
i\hbar \frac{\partial}{\partial t} \hat{U}(t,t,0) \vert \alpha\rangle  =
\hat{H}\hat{U}(t,t_0) \vert \alpha\rangle 
$$
\begin{equation}
  i\hbar \frac{\partial}{\partial t} \vert \alpha,t_0;t\rangle  =
  \hat{H}\vert \alpha,t_0;t\rangle 
  \label{equ:2.16}
\end{equation}
which is the Schrödinger eq. for a state.

The Hamiltonian $\hat{H}$ has in principle an arbitrary form
(it can be an abstract op., or a differential op., or have
matrix form) and be time-dependent or time-independent. We
can straightforwardly obtain the form of $\hat{U}(t,t_0)$
for a time-independent Hamiltonian by solving (\ref{equ:2.15}):
\begin{equation}
  \hat{U}(t,t_0) = \exp(-\frac{i}{\hbar}H(t-t_0))
  \label{equ:2.17}
\end{equation}
which makes sense in general as the series
\begin{equation}
  \hat{U}(t,t_0) = 1 -
  \frac{i}{\hbar}\hat{H}(t-t_0)+\frac{1}{2!}\Big(\frac{-i}{\hbar}\hat{H}(t-t_0)\Big)^2
  \label{equ:2.18}
\end{equation}
A time-dependent Hamiltonian is more complicated. We must
distinguish two cases:
\begin{itemize}
  \item when $[\hat{H}(t_1), \hat{H}(t_2)] = 0$, and
  \item when $[\hat{H}(t_1), \hat{H}(t_2)] \neq 0$
\end{itemize}
In the former case, the solution of (\ref{equ:2.15}) reads
\begin{equation}
  \hat{U}(t,t_0) = \exp\Big(-\frac{i}{\hbar}\int_{t_0}^t dt'
  \, \hat{H}(t')\Big)
  \label{equ:2.19}
\end{equation}
%% todo a partir de aquí saúl volvió a escrbir 2.18
To solve (\ref{equ:2.15}) when $[\hat{H}(t_1), \hat{H}(t_2)]
\neq 0$, we first point out that this differential equation
can be restated as
$$
\hat{U}(t,t_0) \bigg\vert _{t=t_0}^t = 
-\frac{i}{\hbar} \int_{t_0}^t dt' \,
\hat{H}(t')\hat{U}(t',t_0)
$$
recalling that $\hat{U}(t_0,t_0) = 1$ we then have
\begin{equation}
  \hat{U}(t,t_0) = 1 - \frac{i}{\hbar} \int_{t_0}^t dt' \,
  \hat{H}(t')\hat{U}(t',t_0)
  \label{equ:2.19}
\end{equation}
By iteration, we can find the approximate solution
% todo arreglar este desmadre con la numeración
% todo me dí cuenta de que estoy usando mal los bigg
\begin{align}
  \hat{U}(t,t_0) &= 1 - \frac{i}{\hbar} \int_{t_0}^t dt'\,
  \hat{H}(t')\biggr[1 - \frac{i}{\hbar}
  \int_{t_0}^{t'} dt'\, \hat{H}(t'')\hat{U}(t'',t_0)\biggl]
  \nonumber\\
  &=
  1 - \frac{i}{\hbar} \int_{t_0}^t dt' \, \hat{H}(t') + 
  \biggl(\frac{-i}{\hbar}\biggr)^2 \int_{t_0}^t dt' \,
  \int_{t_0}^{t'} dt'' \, \hat{H}(t') \hat{H}(t'')
  \hat{U}(t'',t_0)\nonumber\\
  &=
  1 + \sum_{n=1}^{\infty} 
  \biggl(\frac{-i}{\hbar}\biggr)^n
  \int_{t_0}^t dt_1\,
  \int_{t_0}^{t_1} dt_2\,
  \cdots
  \int_{t_0}^{n-1} dt_n\,
  \hat{H}(t_1) \hat{H}(t_2) \cdots \hat{H}(t_n)
  \label{equ:2.20}
\end{align}
well-known as the Dyson series.
\subsection{Time evolution of stationary energy eigenstates}
Let us suppose that the Hamiltonian is time-independent. We
would like to figure out the evolution of an arbitrary state
$\vert \alpha\rangle $. This is particularly simple if we expand
$\vert \alpha\rangle $ in the basis of eigenstates $\vert a\rangle $ of an operator
$\hat{A}$, such that
\begin{equation}
  [\hat{A}, \hat{H}] = 0
  \label{equ:2.21}
\end{equation}
i.e., compatible with $\hat{H}$. It follows that $\{\vert a\rangle \}$
are energy eigenstates too, whose energy eigenvalues are
\begin{equation}
  \hat{H} \vert a\rangle  = E_a \vert a\rangle 
  \label{equ:2.22}
\end{equation}

Expanding the time-evolution operator in the states $\vert a\rangle $,
we find
%% todo areglar el alig para la numeración
\begin{align}
  \hat{U}(t, 0) = \exp\Big(-\frac{i}{\hbar} \hat{H}t\Big) &= 
  \sum_{a'} \vert a'\rangle \langle a'\vert  \exp(-i \hat{H} t / \hbar) \sum_a
  \vert a\rangle \langle a\vert \nonumber\\
  &=
  \sum_{a,a'} \vert a'\rangle \langle a\vert  \langle a'\vert \exp(-i\hat{H}t/\hbar)
  \vert a\rangle \nonumber\\
  % todo insertar braces
  &=
  \sum_a \exp(-i E_a t /\hbar) \vert a\rangle \langle a\vert 
  \label{equ:2.23}
\end{align}
Now expanding $\vert \alpha\rangle $ in the basis of energy eigenstates,
we see that
\begin{align*}
\vert \alpha\rangle  &= \sum_a \langle a\vert \alpha\rangle  \vert a\rangle  = \sum_a c_a(t=0) \vert a\rangle 
  \nonumber\\
  \Rightarrow \vert \alpha;t\rangle  &= U(t,0) \vert \alpha\rangle  = \sum_{a'}
  \exp\Big(-i E_{a'}t/\hbar\Big) \vert a'\rangle \langle a'\vert a\rangle  c_a(t=0)
  \nonumber\\
  &=
  \sum_a c_a(t= 0) \exp(-iE_a t/\hbar) \vert a\rangle \\
  &\equiv \sum_a c_a(t) \vert a\rangle  \label{equ:2.24}
\end{align*}
where we realize the time evolution of the expansion
coefficients.

Notice that for $\vert \alpha\rangle  = \vert a\rangle $, we find that
\begin{equation}
  \vert a;t\rangle  = \exp(-i E_a t/\hbar) \vert a\rangle 
  \label{equ:2.25}
\end{equation}
so that any simultaneous eigenstate of $H$ and $A$ remains
so at all time.
\subsection{Expectation values}
We are interested in the expectation value of an observable
$B$ in the state
$$
  \vert a;t\rangle  = \hat{U}(t,0) \vert a\rangle 
$$
as given in (\ref{equ:2.26}). The expectation value is
computed as
$$
\langle \hat{B}\rangle  = \langle a\vert \hat{U}^{\dagger}(t,0) \hat{B} \hat{U}(t,0) \vert a\rangle 
= 
\langle a\vert \exp(iE_a t/\hbar) \hat{B} \exp(-i E_a t/\hbar) \vert a\rangle 
$$

Since $U(t,0)$ is a numeric phase, the result is
\begin{equation}
  \langle \hat{B}\rangle  = \langle a;t\vert \hat{B}\vert a;t\rangle  = \langle a\vert \hat{B}\vert a\rangle 
  \label{equ:2.26}
\end{equation}
i.e., expectation values do not change in time for $\vert \alpha\rangle 
= \vert a\rangle $. This is why these states are called stationary.

The nonstationary states $\vert \alpha\rangle  = \sum_a c_a(0) \vert a\rangle $ are
different. In this case, we have
\begin{align*}
  \langle \hat{B}\rangle  &= \langle \alpha;t\vert \hat{B}\vert \alpha;t\rangle  =
  \langle \alpha\vert \hat{U}^{\dagger}(t,0)
  \hat{B}\hat{U}(t,0)\vert \alpha\rangle \\
  &=
  \sum_a \sum_{a'} \langle a\vert c^*_a(0) \exp(i E_a t/\hbar)
  \hat{B}c_{a'}(0)\exp(-i E_{a'} t/\hbar)\vert a'\rangle \\
  &=
  \sum_a \sum_{a'} c_a^*(0) c_{a'}(0) \exp(-i (E_{a'}-E_a)t/
  \hbar) \langle a\vert \hat{B}\vert a'\rangle \\
  &=
  \sum_a \sum_{a'} c_a^*(0) c_{a'}(0) \exp(-i \omega_{a' a}t)
  \langle a\vert \hat{B}\vert a'\rangle 
  % eq 2.27
\end{align*}
where the Bohr's frequency $w_{a'a} =
\frac{1}{\hbar}(E_{a'}-E_a)$ is the oscillation frequency of
the expectation value, and $\langle a\vert \hat{B}\vert a'\rangle $ denote the
matrix elements of the observable $B$ in the basis of $A$.

\subsection{Schrödinger's wave equation}
\subsubsection{Schrödinger's equation in space
representation}
Let us define the position states $\vert \vec{x}\rangle $ as the
eigenstates of the position operator $\hat{\vec{x}}$, such
that
\begin{equation}
  \hat{\vec{x}} \vert \vec{x}'\rangle  = \vec{x}'\vert \vec{x}'\rangle 
  \label{equ:2.28}
\end{equation}
Thier orthonormality is given by
\begin{equation}
  \langle \vec{x}\vert \vec{x}'\rangle  = \delta^3(\vec{x}-\vec{x}')
  \label{equ:2.29}
  % todo al parecer de nuevo hay un desfase
\end{equation}
Expanded in these terms, an arbitrary physical state
$\vert \alpha\rangle $ is given by
$$
  \vert \alpha\rangle  = \int d^3 x \, \vert \vec{x}\rangle \langle \vec{x}\vert \alpha\rangle  \equiv
  \int d^3 \, \psi_{\alpha}(\vec{x}) \vert \vec{x}\rangle 
$$
where $\psi_{\alpha}(\vec{x})$ is usually called the wave
function of the state $\vert \alpha\rangle $ in the representation of
positions. Note that the interpretation of
$\psi_{\alpha}(\vec{x})$ is analogous to that of the
expansion coefficient $c_a = \langle a\vert \alpha\rangle $ of our previous
discussion, i.e., probability amplitud, such that
$\vert \psi_{\alpha}(\vec{x})\vert ^2\, d^3x$ is the probability of
finding in a narrow region $d^3 x$ around $\vec{x}$.

One may be interested in finding the element
$\langle \beta\vert \hat{A}\vert \alpha\rangle $ for some states $\vert \alpha\rangle $,
$\vert \beta\rangle $, in the position basis
\begin{align}
  \langle \beta\vert \hat{A}\vert \alpha\rangle  &= \int d^3 x\, \int d^3 x'\,
  \langle \beta\vert \vec{x}\rangle \langle \vec{x}\vert \hat{A}\vert \vec{x}'\rangle \langle \vec{x}'\vert \alpha\rangle \nonumber\\
  &=\int d^3 x\, \int d^3 x'\,
  \psi_{\beta}^* \langle \vec{x}\vert \hat{A}\vert \vec{x}'\rangle 
  \psi_{\alpha}(\vec{x}')
  \label{equ:2.30}
\end{align}
which depends on the matrix element
$\langle \vec{x}\vert \hat{A}\vert \vec{x}'\rangle $ of $\hat{A}$. This expression
is greatly simplified if $\vec{A}$ is a polynomial of the
operator $\vec{x}$, for example, $\hat{A}$ can take the
form\footnote{As we shall see, this form corresponds to the
potential of the harmonic oscillator.}
in one dimension
\begin{equation}
  \hat{A} = \frac{1}{2} m \omega^2 \vec{x}^2
  \label{equ:2.31}
\end{equation}
In this case, the matrix element in (\ref{equ:2.30}) looks
like
\begin{align*}
  \langle \vec{x}\vert \hat{A}\vert \vec{x}'\rangle  &\mapsto \langle x\vert \hat{A}\vert x'\rangle  =
  \frac{1}{2} m \omega^2 \langle x\vert (\hat{x}^2\vert x'\rangle ) \\
  &=
  \frac{1}{2}m\omega^2 \langle x\vert (\vert x'\rangle x'^2) = \frac{1}{2}m\omega^2
  x'^2\langle x\vert x'\rangle \\
  &=
  \frac{1}{2}m\omega^2 x'^2 \delta(x-x')
  %eq 2.32
\end{align*}
where we have used (\ref{equ:2.28}) and (\ref{equ:2.29}) in
ther 1D versions. This expresion can be generalized for
$\hat{A} = f(\hat{\vec{x}})$ to
\begin{equation}
  \langle \vec{x}\vert f(\hat{\vec{x}})\vert \vec{x}'\rangle  f(\vec{x}')\delta^3
  (\vec{x}-\vec{x}')
  \label{equ:2.33}
\end{equation}
and thus we obtain in this case
\begin{equation}
  \langle \beta\vert \hat{A}\vert \alpha\rangle  = \int d^3x \, \psi_{\beta}^*
  f(\vec{x}) \psi_{\alpha}(\vec{x})
  \label{equ:2.34}
\end{equation}
To arrive at the wave equation proposed by Schrödinger, we
need further consider the representation of the momentum
operator $\hat{\vec{p}}$ in the position basis
\begin{align}
  \langle \vec{x}\vert \hat{\vec{p}}\vert \alpha\rangle  &=
  \int d^3 x' \, \langle \vec{x}\vert \vec{x}'\rangle 
  \langle \vec{x}'\vert \hat{\vec{p}}\vert \alpha\rangle  = 
  \int d^3 x' \, \delta^3(\vec{x}-\vec{x}')
  \int d^3 x'' \, \langle \vec{x}'\vert \hat{\vec{p}}\vert \vec{x}''\rangle 
  \langle \vec{x}''\vert \alpha\rangle  \nonumber\\
  &=
  \int d^3 x'' \, \langle \vec{x}\vert \hat{\vec{p}}\vert \vec{x}''\rangle 
  \psi_{\alpha}(\vec{x}'')\nonumber\\
  &=
  \int d^3 x'' \delta^3(\vec{x}-\vec{x}') \bigl(-i\hbar
  \vec{\nabla}''\bigr) \psi_{\alpha}(\vec{x}'')\nonumber\\ 
  &=
  -i\hbar \vec{\nabla} \psi_{\alpha}(\vec{x})
  \label{equ:2.35}
\end{align}
Let us now consider the Schrödinger equaiton,
(\ref{equ:2.16}),
$$
  i\hbar \frac{\partial}{\partial t} \vert \alpha,t_0;t\rangle  =
  \hat{H}\vert \alpha,t_0;t\rangle 
$$
in the position representation, with a Hamiltonian operator
given by
\begin{equation}
  \hat{H} = \frac{1}{2m} \hat{\vec{p}}^2 +
  \hat{V}(\hat{\vec{x}})
  \label{equ:2.36}
\end{equation}
where $\hat{V}(\hat{\vec{x}})$ is a polynomial function of
$\hat{\vec{x}}$ and, thus, a Hamiltonian operator. From
(\ref{equ:2.35}), we see that
\begin{equation}
  \langle \vec{x}\vert \hat{V}(\hat{\vec{x}})\vert \vec{x}'\rangle  =
  V(\vec{x}'')\delta^3(\vec{x}-\vec{x}')
  \label{equ:2.37}
\end{equation}
Multiplying the Schrödinger equation for states
(\ref{equ:2.16}) by the bra $\langle \vec{x}\vert $ we find
\begin{align}
  i\hbar \frac{\partial}{\partial t} \langle \vec{x}\vert \alpha,t_0;t\rangle 
  &=
  \langle \vec{x}\vert \frac{1}{2m} \hat{\vec{p}}^2 \vert \alpha,t_0;t\rangle  +
  \langle \vec{x}\vert \hat{V}(\hat{\vec{x}})\vert \alpha,t_0;t\rangle \nonumber\\
  &=
  \frac{1}{2m}(-i\hbar \vec{\nabla})^2
  \langle \vec{x}\vert \alpha,t_0;t\rangle  + V(\vec{x}) \langle \vec{x}\vert \alpha,t_0;t\rangle 
  \label{equ:2.38}
\end{align}
where we have used (\ref{equ:2.35}) and (\ref{equ:2.34})
with $\langle \beta\vert  = \langle \vec{x}\vert $. Defining
\begin{equation}
  \psi_{\alpha} (\vec{x},t) \equiv \langle \vec{x}\vert \alpha,t_0;t\rangle ,
  \label{equ:2.39}
\end{equation}
we obtain the time-dependent Schrödinger's wave equation
\begin{equation}
  i\hbar \frac{\partial}{\partial t}
  \psi_{\alpha}(\vec{x},t) =
  -\frac{\hbar^2}{2m} \vec{\nabla}^2 \psi_{\alpha}
  (\vec{x},t) +
  V(\vec{x}) \psi_{\alpha}(\vec{x},t)
  \label{equ:2.40}
\end{equation}
The stationary wave equation is obtained when $\vert \alpha\rangle  =
\vert a\rangle $ is an energy eigenstate and thus stationary. From
(\ref{equ:2.25}) we have that
\begin{align}
  \langle \vec{x}\vert \alpha;t\rangle  = \langle \vec{x}\vert a;t\rangle  = \exp(-i E_a t/\hbar)
  \langle \vec{x}\vert a\rangle  = \exp(-i E_{\alpha}t /\hbar)
  \langle \vec{x}\vert \alpha\rangle \nonumber\\
  \Leftrightarrow \psi_{\alpha}(\vec{x},t) = \exp(-i
  E_{\alpha} t/\hbar) \psi_{\alpha}(\vec{x})
  \label{equ:2.41}
\end{align}
where we have defined $\psi_{\alpha}(\vec{x}) \equiv
\langle \vec{x}\vert \alpha\rangle $ in analogy with (\ref{equ:2.39}).
It then follows that (\ref{equ:2.40}) becomes
\begin{equation}
  E_{\alpha} \psi_{\alpha} (\vec{x}) = - \frac{\hbar^2}{2m}
  \vec{\nabla}^2 \psi_{\alpha} (\vec{x}) + V(\vec{x})
  \psi_{\alpha} (\vec{x})
  \label{equ:2.42}
\end{equation}
when the time dependence carried by the time-evolution
operator has been factorized and eliminated.
\subsubsection{Schrödinger's equation in momentum
representation}
As in the configuration space, we can now define the
eigenstates of the momentum operator $\hat{\vec{p}}$ as
\begin{equation}
  \hat{\vec{p}} \vert \vec{p}'\rangle  = \vec{p}' \vert \vec{p}'\rangle ,
  \label{equ:2.43}
\end{equation}
satisfying the orthonormality condition
\begin{equation}
  \langle \vec{p}\vert \vec{p}'\rangle  = \delta^3(\vec{p}-\vec{p}')
  \label{equ:2.44}
\end{equation}
and the completeness relation
\begin{equation}
  1 = \int d^3 p \vert \vec{p}\rangle \langle \vec{p}\vert 
  \label{equ:2.45}
\end{equation}
An arbitrary state in the basis reads
\begin{equation}
  \vert \alpha\rangle  = \int d^3\, p \vert \vec{p}\rangle \langle \vec{p}\vert \alpha\rangle  = \int d^3 \vert ,
  \phi_{\alpha(\vec{p})}\vert \vec{p}\rangle 
  \label{equ:2.46}
\end{equation}
where we have defined the momentum-space wave function
\begin{equation}
  \phi_{\alpha} (\vec{p}) \equiv \langle \vec{p}\vert \alpha\rangle 
  \label{equ:2.47}
\end{equation}
normalized as
\begin{equation}
  \int d^3p \, \vert \phi_{\alpha}(\vec{p})\vert ^2 = \int d^3p \,
  \langle \alpha\vert \vec{p}\rangle \langle \vec{p}\vert \alpha\rangle  = \langle \alpha\vert \alpha\rangle  = 1
  \label{equ:2.48}
\end{equation}
if $\vert \alpha\rangle $ is correctly normalized.
The wave function $\phi_{\alpha} (\vec{p})$ is related to
$\psi_{\alpha}(\vec{x})$ by
\begin{equation}
  \phi_{\alpha}(\vec{p}) = \langle \vec{p}\vert \alpha\rangle  = \int d^3x \,
  \langle \vec{p}\vert \vec{x}\rangle \langle \vec{x}\vert \alpha\rangle  = \int d^3 x\,
  \langle \vec{p}\vert \vec{x}\rangle \psi_{\alpha}(\vec{x})
  \label{equ:2.49}
\end{equation}
and conversely
\begin{equation}
  \psi_{\alpha}(\vec{x}) = \int d^3 x\, \langle \vec{x}\vert \vec{p}\rangle 
  \phi_{\alpha}(\vec{p})
  \label{equ:2.50}
\end{equation}
where $\langle \vec{p}\vert \vec{x}\rangle  = \langle \vec{x}\vert \vec{p}\rangle ^*$ is yet
unknown. To figure out $\langle \vec{p}\vert \vec{x}\rangle $, let us compute
\begin{align}
  \langle \vec{x}\vert \hat{\vec{p}}\vert \vec{p}\rangle  &= \int d^3 x' \,
  \langle \vec{x}\vert \hat{\vec{p}}\vert \vec{x}'\rangle 
  \langle \vec{x}'\vert \vec{p}\rangle \nonumber\\
  &= -i\hbar \vec{\nabla} \langle \vec{x}\vert \vec{p}\rangle  \label{equ:2.51}
\end{align}
which, using (\ref{equ:2.43}), implies
\begin{equation}
  \vec{p} \langle \vec{x}\vert \vec{p}\rangle  = -i \hbar
  \vec{\nabla}\langle \vec{x}\vert \vec{p}\rangle 
  \label{equ:2.52}
\end{equation}
whose solution reads
\begin{equation}
  \langle \vec{x}\vert \vec{p}\rangle  = N \exp(i \vec{p} \cdot \vec{x}/\hbar)
  \label{equ:2.53}
\end{equation}

The normalization factor can easily be found by computing
\begin{align}
  \langle \vec{x}\vert \vec{x}'\rangle  &= \int d^3 p \, \langle \vec{x}\vert \vec{p}\rangle 
  \langle \vec{p}\vert \vec{x}'\rangle \nonumber\\
  &= N^2 \int d^3 p\, \exp\Bigl(-i \vec{p}\cdot
  (\vec{x}-\vec{x}')/\hbar\Bigr) \nonumber\\
  &=
  N^2 (2\pi \hbar)^3 \delta^3(\vec{x}-\vec{x}')
  \label{equ:2.54}
\end{align}
where the last step is the result of realizing that the
integral is just a Fourier transform. Notice that the l.h.s.
is actually just $\delta^3(\vec{x}-\vec{x}')$, which finally
lead to
\begin{equation}
  N^2 = (2\pi\hbar)^{-3}
  \label{equ:2.55}
\end{equation}
Replacing into (\ref{equ:2.53}), we find that
\begin{equation}
  \langle \vec{x}\vert \vec{p}\rangle  = \frac{1}{(2\pi\hbar)^{3/2}} \exp(i
  \vec{p}\cdot \vec{x}/\hbar)
  \label{equ:2.56}
\end{equation}
It then follows from (\ref{equ:2.49}) and (\ref{equ:2.50}) that
$\phi_{\alpha}(\vec{p})$ is just the Fourier transform of
$\psi_{\alpha}(\vec{x})$, and vice versa
\begin{align}
  \psi_{\alpha}(\vec{x}) &= \int d^3x \,
  \frac{1}{2\pi\hbar}^{3/2}
  \exp\Bigl(i\vec{p}\cdot\vec{x}/\hbar\Bigr)
  \phi_{\alpha(\vec{p})}\\
  % todo creo que aquí va un error
  \phi_{\alpha}(\vec{p}) &= \int d^3x \,
  \frac{1}{2\pi\hbar}^{3/2}
  \exp\Bigl(-i\vec{p}\cdot\vec{x}/\hbar\Bigr)
  \psi_{\alpha}(\vec{x})
\end{align}
Let us use these results to find the Schrödinger's wave
equation in momentum space for the Hamiltonian
(\ref{equ:2.36}), defining additionally
\begin{equation}
  \phi_{\alpha}(\vec{p},t) \equiv \langle \vec{p}\vert \alpha,t_0;t\rangle 
  \label{equ:2.58}
\end{equation}

We have

\begin{align}
  i\hbar \frac{\partial}{\partial t} \langle \vec{p}\vert \alpha,t_0;t\rangle 
  &= \langle \vec{p}\vert \frac{1}{2m}\hat{\vec{p}}^2\vert \alpha,t_0;t\rangle 
  \nonumber \\ &+
  \int d^3 p' \, \langle \vec{p}\vert \hat{V}(\hat{\vec{x}}) \vert \vec{p}'\rangle 
  \langle \vec{p}'\vert \alpha,t_0;t\rangle  \nonumber \\
  &=
  \frac{1}{2m} \vec{p}^2 \phi_{\alpha}(\vec{p},t) +
  \int d^3 p' \,
  \int d^3 x\,
  \int d^3 x'\,
  \langle \vec{p}\vert \vec{x}\rangle 
  \langle \vec{x}\vert \hat{V}(\hat{\vec{x}})\vert \vec{x}'\rangle \langle \vec{x}'\vert \vec{p}'\rangle 
  \phi_{\alpha}(\vec{p},t)\nonumber\\
  &=
  \frac{\vec{p}^2}{2m} \phi_{\alpha}(\vec{p},t) \nonumber\\&+
  \frac{1}{(2\phi\hbar)^3}
  \int d^3 p' \,
  \int d^3 x\,
  \int d^3 x'\,
  \exp(-i\vec{p}\cdot\vec{x}/\hbar)
  V(\vec{x})\delta^3(\vec{x}-\vec{x}')
  \exp(i\vec{p}'\cdot\vec{x}'/\hbar)\phi_{\alpha}(\vec{p},t)\nonumber\\
  &= \frac{\vec{p}^2}{2m} \phi_{\alpha}(\vec{p},t)
  \nonumber\\&+
  \frac{1}{(2\pi\hbar)^{3/2}}
  \int d^3 p' \,
  \int \frac{d^3x}{(2\pi\hbar)^{3/2}}
  \exp(-i(\vec{p}-\vec{p}')\cdot\vec{x}/\hbar) V(\vec{x})
  \phi(\vec{p},t)\nonumber
  \intertext{which finally yields}
  i\hbar \frac{\partial}{\partial t} \phi_{\alpha}
  (\vec{p},t) &=
  \frac{\vec{p}^2}{2m} \phi_{\alpha} + 
  \frac{1}{(2\pi\hbar)^{3/2}}
  \int d^3 p' \,
  V(\vec{p}-\vec{p}') \phi_{\alpha} (\vec{p},t)
  \label{equ:2.59}
\end{align}

where

\begin{equation}
  V(\vec{p}-\vec{p}') \equiv \frac{1}{(2\pi\hbar)^{3/2}}
  \int \, \exp(-i(\vec{p}-\vec{p}'')\cdot \vec{x}/\hbar)
  V(\vec{x}) \, d^3 x
  \label{equ:2.60}
\end{equation}
\subsubsection{Continuity equation}
If we let the probability density of a system to be defined
as
\begin{equation}
  \rho \equiv \psi^*_{\alpha}(\vec{x},t)
  \psi_{\alpha}(\vec{x},t)
  \label{equ:2.61}
\end{equation}
then one easily finds that (omitting for simplicity the
dependence on $\vec{x}$ \& $t$)
\begin{align}
  \frac{\partial \rho}{\partial t} &= \psi_{\alpha}^*
  \frac{\partial}{\partial t} \psi_{\alpha} +
  \frac{\partial}{\partial t} \psi_{\alpha}^* \psi_{\alpha}
  \nonumber \\
  &=
  \psi_{\alpha}^* \frac{1}{i\hbar}
  \biggl(-\frac{\hbar^2}{2m}\vec{\nabla}^2 \psi_{\alpha} +
  V\psi_{\alpha}\biggr) - 
  \frac{1}{i\hbar} \biggl(-\frac{\hbar^2}{2m}\vec{\nabla}^2
  \psi_{\alpha}^* + V\psi_{\alpha}^* \biggr) \psi_{\alpha}
  \nonumber \\
  &=
  \frac{i\hbar}{2m} \biggl[\psi_{\alpha}^* \vec{\nabla}^2
  \psi_{\alpha}-\bigl(\vec{\nabla}^2
\psi_{\alpha}^*\bigr)\psi_{\alpha}\biggr] \nonumber \\
&=
%utilizar cancel en la siguiente línea %todo
\frac{i\hbar}{2m}
\biggl[
  \vec{\nabla}\bigl(\psi_{\alpha}^*\cdot
  \vec{\nabla}\psi_{\alpha}\bigr)-
  \vec{\nabla}\psi_{\alpha}^* \cdot
  \vec{\nabla}\psi_{\alpha} -
  \vec{\nabla}\cdot\bigl((\vec{\nabla}\psi_{\alpha}^*)\psi_{\alpha}\bigr)
  +
  \vec{\nabla}\psi_{\alpha}^* \cdot \vec{\nabla}\psi_{\alpha}
\biggr]\nonumber \\
&=
\frac{i\hbar}{2m} \vec{\nabla}\cdot \biggl(\psi_{\alpha}^* 
\vec{\nabla}\psi_{\alpha} -
\bigl(\vec{\nabla}\psi_{\alpha}^*\bigr)\psi_{\alpha}
\biggr) \nonumber\\
&=
- \vec{\nabla} \vec{j} \label{equ:2.62}
\end{align}
where
\begin{equation}
  \vec{j}\equiv
\biggl(\psi_{\alpha}^* 
\vec{\nabla}\psi_{\alpha} -
\bigl(\vec{\nabla}\psi_{\alpha}^*\bigr)\psi_{\alpha}
\biggr) \frac{i\hbar}{2m}
  \label{equ:2.63}
\end{equation}
Rewriting (\ref{equ:2.62}) in these terms, we find
\begin{equation}
  \frac{\partial \rho}{\partial t} + \vec{\nabla}\cdot
  \vec{j} = 0
  \label{equ:2.64}
\end{equation}
which coincides with the well-known continuity equation for
a probability density $\rho$ and a probability current
$\vec{j}$, defined respectively by (\ref{equ:2.61}) \&
(\ref{equ:2.63})
% new pages
It's interesting to realize that any continuity relation
reveals a conserved quantity. In this case, the proability
\begin{equation}
  P = \int d^3x \, \rho
  \label{equ:2.65}
\end{equation}
is conserved (as expected):
\begin{equation}
  \frac{dP}{dt} = \int_{V} d^3 x \, \frac{\partial
  p}{\partial t} = - \int_{\partial V} d\vec{S} \, \cdot
  \vec{j} = 0
  \label{equ:2.66}
\end{equation}
which vanishes on ly if (\ref{equ:2.63}) vanishes at the
boundary $\partial V$ located infinitely far away. This
condition is satisfied because \(\displaystyle \lim_{\vec{x}
\to \infty} \psi_{\alpha}(\vec{x},t)  = 0\) is a condition
of square integrability, i.e., for $P$ in (\ref{equ:2.65})
to be finite always.
\subsection{Elementary 1 quantum systems}
\subsubsection{Harmonic oscillator in operator method}
The harmonic oscillator is one of the most important quantum
systems because any physical system is some approximation
(around its stability point) resembles an harmonic
oscillator. The potential of this system is give, as in
classical mechanics, by 
\begin{equation}
  \hat{V}(\hat{x}) = \frac{1}{2} m \omega^2 \hat{x}^2
  \label{equ:2.67}
\end{equation}
We would like to find the eigenstates of the Hamiltonian as
a time-independent system. To do so, we shall follow the
elegant procedure envisages by Heisenberg based on the
non-hermitian ``ladder operators"
\begin{subequations}
  \begin{align}
    \hat{a}&=\sqrt{\frac{m\omega}{2\hbar}}\Bigl(\hat{x}+\frac{i\hat{p}}{m\omega}\Bigr)\label{equ:2.68a}\\
    \hat{a}^{\dagger}&=\sqrt{\frac{m\omega}{2\hbar}}\Bigl(\hat{x}-\frac{i\hat{p}}{m\omega}\Bigr)\label{equ:2.68b}
  \end{align}
  \label{equ:2.68}
\end{subequations}
which let one identify the operators $\hat{x}$ ans $\hat{p}$
as
\begin{subequations}
  \begin{align}
    \hat{x} &=
    \sqrt{\frac{\hbar}{2m\omega}}(\hat{a}+\hat{a}^{\dagger})\label{equ:2.69a}\\
    \hat{p} &=
    i\sqrt{\frac{m\hbar \omega}{2}}(\hat{a}^{\dagger}-\hat{a})\label{equ:2.69b}
  \end{align}
\end{subequations}
% finish page 14
The ladder operators do not commute:
\begin{align}
  [\hat{a}, \hat{a}^{\dagger}] &= \frac{m\omega}{2\hbar}
  \biggl([\hat{x},-\frac{i\hat{p}}{m\omega}]+[-\frac{i\hat{p}}{m\omega},\hat{x}]\biggr)
  \nonumber\\
  &= \frac{i}{2\hbar}\biggl(
  -[\hat{x},\hat{p}]+[\hat{p},\hat{x}]
  \biggr)\label{equ:2.70}
\end{align}
Defining the so-called ``number operator"
\begin{equation}
  \hat{N} \equiv \hat{a}\hat{a}^{\dagger}
  \label{equ:2.71}
\end{equation}
it is easy to realize that in terms of $\hat{x}$ and
$\hat{p}$
\begin{align}
  \hat{N} &= \frac{m\omega}{2\hbar} \biggl(\hat{x}^2 +
  \frac{\hat{p}^2}{m^2 \omega^2}\biggr) +
  \frac{i}{2\hbar}[\hat{x},  \hat{p}] \nonumber \\
  &=
  \frac{1}{\hbar\omega}\biggl(\frac{\hat{p}^2}{2m}+\frac{1}{2}m\omega^2
  \hat{x}^2\biggr) - \frac{1}{2} \label{equ:2.72}
  % poner unos braces aquí
\end{align}
Therefore, it follows that the Hamiltonian can be expressed
in terms of the number operator as
\begin{equation}
  \hat{H} = \hbar\omega \Bigl(\hat{N}+1/2\Bigr)
  \label{equ:2.73}
\end{equation}
Note that $\hat{H}$ and $\hat{N}$ are compatible (and
hermitian) operators because
$$
[\hat{H},  \hat{N}] = 0;
$$
they can thus be simultaneously diagonalized, so that the
eigenstates of $\hat{N}$ are also energy eigenstates.
The eigenstates of $\hat{N}$ are defined by
\begin{equation}
  \hat{N} \vert n\rangle = n \vert n\rangle 
  \label{equ:2.74}
\end{equation}
where $n$ i san arbitrary real number at this point. It thn
follows that
\begin{equation}
  \hat{H} \vert n\rangle  = \hbar\omega (n+1/2)\vert n\rangle 
  \label{equ:2.75}
\end{equation}
% finished page 15
Inserting (\ref{equ:2.75}) in the time-independent
Schrödinger's equation, we identify
$$
E_n = \hbar\omega (n+1/2)
$$
as the energy asociated with the energy eigenstate $\vert n\rangle $.
We can determine the values that $n$ can take by exploring
the action of the ladder operators on $\vert n\rangle $. Note first that
\begin{align}
  [\hat{N}, \hat{a}^{\dagger}] &=
  [\hat{a}^{\dagger}\hat{a},\hat{a}^{\dagger}] =
  \hat{a}^{\dagger}[\hat{a}, \hat{a}^{\dagger}] =
  \hat{a}^{\dagger} \nonumber\\
  [\hat{N}, \hat{a}] &= [\hat{a}^{\dagger}\hat{a}, \hat{a}]
  = [\hat{a}^{\dagger}, \hat{a}] \hat{a}=  - \hat{a}
  \label{equ:2.76}
\end{align}
where (\ref{equ:2.70}) has been used. Applying these
results, we readily see that
\begin{subequations}
  \begin{align}
    \hat{N} (\hat{a}^{\dagger} \vert n\rangle ) &=
    \Bigl([\hat{N},\hat{a}^{\dagger}]+\hat{a}^{\dagger}\hat{N}\Bigr)\vert n\rangle 
    = (n+1)(\hat{a}^{\dagger} \vert n\rangle ) \label{equ:2.77a} \\
    \hat{N}(\hat{a}\vert n\rangle ) &= \Bigl([\hat{N},
  \hat{a}]+\hat{a}\hat{N}\Bigr) \vert n\rangle  = (n-1) (\hat{a}\vert n\rangle )
  \label{equ:2.77b}
  \end{align}
\end{subequations}
Eq (\ref{equ:2.77a}) implies that $\hat{a}^{\dagger}\vert n\rangle $ has
the same eigenvalues as the state $\vert n+1\rangle $ or, in other
words, that the action of the operator $\hat{a}^{\dagger}$
raises a state from $\vert n\rangle $ to $\vert n+1\rangle $. Analogously, $\hat{a}$
lowers the state from $\vert n\rangle $ to $\vert n-1\rangle $. This is why the
ladder operators $\hat{a}$ and $\hat{a}^{\dagger}$ are known
as lowering and rising operators, respectively.
It immediatly folows that the following relations must hold
\begin{align}
  \hat{a}\vert n\rangle  &= c \vert n-1\rangle  \nonumber\\
  \hat{a}^{\dagger}\vert n\rangle  &= d \vert n+1\rangle  \label{equ:2.78}
\end{align}
where $c$ and $d$ are numbers that may be determinated by
imposing the orthogonality (we shall impose orthonormality)
of the eigenstates of the Hermitian operator $\hat{N}$. The
norm of $\hat{a}\vert n\rangle $ is computed as
$$
\vert c\vert ^2 \langle n-1\vert n-1\rangle  = \langle n\vert \hat{a}^{\dagger}\hat{a}\vert n\rangle  = n\langle n\vert n\rangle ,
$$
which implies
\begin{equation}
  \vert c\vert ^2 = n
  \label{equ:2.79}
\end{equation}
%% pp 17
for $\langle n\vert m\rangle  = \delta_{n,m}$. Note that we can further assume
that $c$ (and $d$) is real because
$$
a^{\dagger}a \vert n\rangle  = a^{\dagger} c_n \vert  n-1\rangle  = c_n d_{n-1}\vert n\rangle  =
n\vert n\rangle 
$$
and $n$ is real. Further we assume by convention that $c$
(and $d$) is positive. This yield
\begin{equation}
  c = \sqrt{n}
  \label{equ:2.80}
\end{equation}
Similarly, normalizing $\hat{a}^{\dagger} \vert n\rangle $ we find
$$
\vert d\vert ^2 \langle n+1\vert n+1\rangle  = \langle n\vert \hat{a}\hat{a}^{\dagger}\vert n\rangle  =
\langle n\vert [\hat{a}, \hat{a}^{\dagger}] + \hat{a}^{\dagger}\hat{a} \vert 
n\rangle   = (n+1) \langle n\vert n\rangle 
$$
which yields
\begin{equation}
  d = \sqrt{n+1}
  \label{equ:2.81}
\end{equation}
We have therefore obtained
\begin{subequations}
  \begin{align}
    \hat{a}\vert n\rangle  &= \sqrt{n}\vert n-1\rangle  \label{equ:2.82a} \\
    \hat{a}^{\dagger}\vert n\rangle  &= \sqrt{n+1}\vert n+1\rangle  \label{equ:2.82b}
  \end{align}
\end{subequations}
with $n \geq 0$ to provide reasonable results. Since the
smallest value that $n$ can take is $n = 0$ and all states
created by $\hat{a}^{\dagger}$ differ by a unit of $n$, $n$
must be integer. Therefore, eq.~(\ref{equ:2.75}) can be
interpreted as a rule of energy quantization.
Notice that the sate of lowert energy satisfies
\begin{equation}
  \hat{a}\vert 0\rangle  = 0
  \label{equ:2.83}
\end{equation}
i.e., it is anihilated bythe lowering operator $\hat{a}$.
Eq.~(\ref{equ:2.83}) can be considered a definition of the
ground state. Note that the ground state has the
non-vanishing energy $\displaystyle{\frac{1}{2}\hbar
\omega}$.
We can now obtain an arbitrary state $\vert n\rangle $ by the repeated
action of $\hat{a}^{\dagger}$ on $\vert 0\rangle $. We see that
\begin{equation}
  \vert 1\rangle  = \hat{a}^{\dagger} \vert 0\rangle 
  \label{equ:2.84}
\end{equation}
and then, from (\ref{equ:2.82b})
\begin{subequations}
  \begin{alignat}{2}
    \vert 2\rangle  &= \frac{\hat{a}^{\dagger}}{\sqrt{2}} \vert 1\rangle  &&=
    \frac{(\hat{a}^{\dagger})^2}{\sqrt{2}} \vert 0\rangle 
    \label{equ:2.85a} \\
    \vert 3\rangle  &= \frac{\hat{a}^{\dagger}}{\sqrt{3}} \vert 2\rangle  &&=
    \frac{(\hat{a}^{\dagger})^3}{\sqrt{3!}} \vert 0\rangle 
    \label{equ:2.85b}
  \end{alignat}
\end{subequations}
and so on. We find the general expression
\begin{equation}
  \vert n\rangle  = \frac{(\hat{a}^{\dagger})^n}{\sqrt{n!}} \vert 0\rangle 
  \label{equ:2.86}
\end{equation}
A useful observation is that the states $\{\vert n\rangle \}$ are not
eigenstates of $\hat{x}$ and $\hat{p}$, as expected, because
$[\hat{H},\hat{x}]$ and $[\hat{H}, \hat{p}]$ do not vanish,
in general.

We can explicitly show that neither $\hat{x}$ nor $\hat{p}$
are diagonal in the basis of energy eigenstates:

\begin{subequations}
  \begin{alignat}{3}
    \langle m\vert \hat{x}\vert n\rangle  &= \sqrt{\frac{\hbar}{2m\omega}}
    \langle m\vert \hat{a}+ \hat{a}^{\dagger}\vert n\rangle  &&=
    \sqrt{\frac{\hbar}{2m\omega}} 
    \bigl[\sqrt{n}\langle m\vert n-1\rangle  + \sqrt{n+1}\langle m\vert n+1\rangle \bigr]
    \nonumber \\
    &{}&&= \sqrt{\frac{\hbar}{2m\omega}}
    \bigl[\sqrt{n}\delta_{m , n-1} + \sqrt{n+1}\delta_{m,
    n+1}\bigr] \label{equ:2.87a} \\
    \langle m\vert \hat{p}\vert n\rangle  &= i \sqrt{\frac{m\hbar \omega}{2}}
    \langle m\vert \hat{a}^{\dagger} - \hat{a}\vert n\rangle  &&= i
    \sqrt{\frac{m\hbar\omega}{2}}
    \bigl[
      \sqrt{n+1}\delta_{m,n+1} - \sqrt{n}\delta_{m,n-1}
    \bigr]\label{equ:2.87b}
  \end{alignat}
\end{subequations}
where the expresions (\ref{equ:2.68}) have been used.
With this formalism, we can compute the energy wavefunctions
in the position representation. Let us start by obtaining
$\psi_0 \equiv \langle x\vert 0\rangle $. To do so, recall that the ground
state is anihilated by $\hat{a}$, so that
\begin{equation}
  \langle x\vert \hat{a}\vert 0\rangle  = \sqrt{\frac{m\omega}{2\hbar}} \langle x\vert \hat{x}+
  \frac{i\hat{p}}{m\omega} \vert 0\rangle  = 0
  \label{equ:2.88}
\end{equation}
Inserting a unity, we have
\begin{equation}
  \int dx' \, \langle x\vert \hat{x} + \frac{i\hat{p}}{m\omega}\vert x'\rangle 
  \langle x'\vert 0\rangle  =
  \int dx' \, \delta(x - x')
  \Bigl(x + \frac{i}{m\omega}\bigl(-i\hbar
  \frac{d}{dx}\bigr)\Bigr)\psi_0 = 0
  \label{equ:2.89}
\end{equation}
which, after integration and rearranging the terms, yields
\begin{equation}
  \frac{d\psi_0(x)}{dx} = -\frac{m\omega}{\hbar} x \psi_0(x)
  \label{equ:2.90}
\end{equation}
From (\ref{equ:2.90}), we realize that
$$
\psi_0(x) = A_0 \exp(-\frac{1}{2}\frac{m\omega}{\hbar}x^2)
$$
or in terms of the dimensionless coordinate $\tilde{x} =
\sqrt{\frac{m\omega}{\hbar}}x$
\begin{equation}
  \psi_0(\tilde{x}) = A_0 \exp(-\frac{1}{2}\tilde{x}^2)
  \label{equ:2.91}
\end{equation}
$A_0$ can be found by normalizing
\begin{equation}
  \int dx \, \psi_0(\tilde{x}) \psi_0(\tilde{x}) = A_0
  \sqrt{\frac{\hbar}{m\omega}} \int d\tilde{x} \,
  \exp(-\tilde{x}^2) = A_0^2 \sqrt{\frac{\hbar\pi}{m\omega}}
  = 1, 
  \label{equ:2.92}
\end{equation}
which leads finally to
\begin{equation}
  \psi_0 (\tilde{x}) =
  \bigl(\frac{\hbar\pi}{m\omega}\bigr)^{-1/4}
  \exp(-\frac{1}{2}\tilde{x}^2)
  \label{equ:2.93}
\end{equation}
$\psi_n(\tilde{x})$ can be obtained from inspecting $\langle x\vert n\rangle $:
%todo arreglar el desastre de abajo
  \begin{alignat}{2}
    \langle x\vert 1\rangle  &= \langle x\vert \hat{a}^{\dagger}\vert 0\rangle  =
    \sqrt{\frac{m\omega}{2\hbar}}
    \langle x\vert \hat{x}-\frac{i\hat{p}}{m\omega}\vert 0\rangle  =
    \sqrt{\frac{m\omega}{2\hbar}}\bigl(x -
    \frac{\hbar}{m\omega}\frac{d}{dx}\bigr) \psi_0(x)
    \nonumber \\
    \langle x\vert 2\rangle  &= \langle x\vert \frac{(\hat{a})^2}{\sqrt{2}}\vert 0\rangle  =
    \frac{1}{\sqrt{2}}\bigl(\sqrt{{m\omega}{2\hbar}}\bigr)^2\bigl(x
      - \frac{\hbar}{m\omega}\bigr)^2 \psi_0(x) \nonumber \\
      \vdots\nonumber \\
      \langle x\vert n\rangle  &= 
      \langle x\vert \frac{(\hat{a})^n}{\sqrt{n!}}\vert 0\rangle  =
      \frac{1}{\sqrt{n!2^n}}
      \frac{1}{\bigl(\frac{\hbar}{m\omega}\bigr)^n}
      \Bigl(x - \frac{\hbar}{m\omega}\frac{d}{dx}\Bigr)^n \psi_0(x)
     \label{equ:2.94}
  \end{alignat}
The latter can be rewritten in terms of $\tilde{x}$ as
\begin{equation}
  \psi_0(\tilde{x}) =
  \frac{1}{\sqrt{n!2^n}}
  \frac{1}{\bigl(\frac{\hbar}{m\omega}\bigr)^n}
  \biggl(
  \sqrt{\frac{\hbar}{m\omega}\tilde{x}-\frac{\hbar}{m\omega}}
  \biggr)^n \psi_0(\tilde{x})
  \label{equ:2.95}
\end{equation}
that is,
\begin{equation}
  \psi_n(\tilde{x}) = \frac{1}{\sqrt{n!2^n}}
  \bigl(
  \frac{\hbar\pi}{m\omega}
  \bigr)^{-1/4}
  \bigl(\tilde{x}-\frac{d}{d\tilde{x}}\bigr)^n
  \exp(-\frac{1}{2}\tilde{x}^2)
  \label{equ:2.96}
\end{equation}
Interestingly, the Rodrigues' formula for the Hermite
polynomials reads\footnote{See e.g. Arfken 1985, p.270}
\begin{equation}
  H_n(\tilde{x}) = \exp\Bigl(\frac{1}{2}\tilde{x}^2\Bigr)
  \Bigl(\tilde{x}-\frac{d}{d\tilde{x}}\Bigr)^n
  \exp\Bigl(-\frac{1}{2}\tilde{x}^2\Bigr)
  \label{equ:2.97}
\end{equation}
% todo arreglar paréntesis mal puestos por todos lados
%pp 20
Therefore, (\ref{equ:2.96}) can be rewritten as
\begin{equation}
  \psi_n(\tilde{x}) = \frac{1}{\sqrt{n!2^n}}
  \Bigl(
    \frac{\hbar \pi}{m\omega}
    \Bigr)^{-1/4}
    \exp\Bigl(-\frac{1}{2}\tilde{x}^2\Bigr)
    H_n(\tilde{x})
  \label{equ:2.98}
\end{equation}
A few observations are in order. Firstly, as depected, the
wavefunctions are eigher symmetric or antisymmetric, but the
probability densities $\vert \psi_n\vert ^2$ are always symmetric on
$\tilde{x}$ (or $x$). It then immediately follows that
$\langle \hat{x}\rangle  = 0$. This result can be easily verified in the
operator formalism, following our result (\ref{equ:2.87a})
\begin{equation}
  \langle \hat{x}\rangle  = \langle n\vert \hat{x}\vert n\rangle  = \sqrt{\frac{\hbar}{2m\omega}}
  \Bigl(
  \sqrt{n}\delta_{n, n-1} + \sqrt{n+1}\delta_{n, n+1}
  \Bigr)
  = 0
  \label{equ:2.99}
\end{equation}
Similarly, from (\ref{equ:2.87b}), we find that
\begin{equation}
  \langle \hat{p}\rangle  = p
  \label{equ:2.100}
\end{equation}
Furthermore, we can compute the expectation values of
$\hat{x}^2$ \& $\hat{p}^2$ as follows

\begin{subequations}
  \begin{alignat}{4}
    \langle \hat{x}^2\rangle  &= \frac{\hbar}{2m\omega}
    \langle n\vert \bigl(\hat{a}+\hat{a}^{\dagger}\bigr)\vert n\rangle  &&=
    \frac{\hbar}{2m\omega}\langle n\vert \bigl(\hat{a}^2 +
    \hat{a}\hat{a}^{\dagger} + \hat{a}^{\dagger}\hat{a} +
    \hat{a}^{\dagger}{}^2\bigr)\vert n\rangle  \nonumber \\
    &{} &&= \frac{\hbar}{2m\omega}\langle n\vert (1 +
    2\hat{a}\hat{a}^{\dagger})\vert n\rangle  =
    \frac{\hbar}{2m\omega}(1+2n) \label{equ:2.101}
  \end{alignat}
\end{subequations}
where we used that $\langle n\vert \hat{a}^2\vert n\rangle  =
\langle n\vert (\hat{a}^{\dagger})^2 \vert  n\rangle  = 0$ and (\ref{equ:2.70}).
Similarly,
\begin{equation}
  \langle \hat{p}{}^2\rangle  = - \frac{m\hbar\omega}{2}
  \langle n\vert (\hat{a}{}^{\dagger}-\hat{a})^2 \vert n\rangle  =
  -\frac{m\hbar\omega}{2}
  \langle n\vert (-\hat{a}\hat{a}{}^{\dagger}-\hat{a}\hat{a}^{\dagger})\vert n\rangle 
  =
  \frac{m\hbar\omega}{2}(1+2n)
  \label{equ:2.102}
\end{equation}
Noticing now that $\langle (\Delta \hat{x})^2 \rangle  = \langle \hat{x}^2\rangle $
because of (\ref{equ:2.99}), and $\langle (\Delta \hat{p})^2\rangle  =
\langle \hat{p}^2\rangle $ because of (\ref{equ:2.100}), we find the
uncertainty relation
\begin{equation}
  \langle (\Delta \hat{x})^2 \rangle  \langle  (\Delta \hat{p})^2\rangle  =
  \frac{\hbar^2}{4}(1+2n)^2,
  \label{equ:2.102}
\end{equation}
%%pp 20
which satisfies the uncertainty principle. We notice that
the state of minimal uncertainty is precisely the ground
state, which, as given by (\ref{equ:2.93}), has a Gaussian
profile that are typically considered maximally coherent.
A second observation regards the natural question of how
these states evolve in time. In section (2.1),
%todo label sections
we learned that stationary states, such as $\vert n\rangle $ evolve
trivially as
\begin{equation}
  \vert n,t\rangle  = \exp\Bigl(-\frac{i}{\hbar}\hat{H}t\Bigr) \vert n\rangle  = \exp(-i\omega
  (n+1/2)t)\vert n\rangle 
  \label{equ:2.103}
\end{equation}
For these states, one can easely verify that
\begin{equation}
  \langle n;t\vert \hat{x}\vert n;t\rangle  = \langle n\vert \hat{a}\vert n\rangle  = 0 =
  \langle n\vert \hat{p}\vert n\rangle  = \langle n;t\vert \hat{p}\vert n;t\rangle 
  \label{equ:2.104}
\end{equation}
and, thus, the uncertainty relation (\ref{equ:2.102}) holds
still. A natural puzzle is that $\langle \hat{x}\rangle  = 0$ even through
the system we analize is a harmonic oscillator. From
classical mechanics, one may have expected on oscillatory
behavior. How come that this fails quantum-mechanically? The
answer is trivial: we have considered only stationary
states, which may not capture all features of the system.
Non-stationary states fo the form
\begin{equation}
  \vert \alpha\rangle  = \sum_n c_n \vert n\rangle 
  \label{equ:2.105}
\end{equation}
as discussed around (\ref{equ:2.27}), have a non-trivial
evolution even if the Hamiltonian is time-independent. We
shall shortly see that these is a special kind of
non-stationary states that keep the nature of the classical
harmonic oscillator, the so-called coherent states.
A second observation conserns the method we have used to
determine the features of the quantum harmonic oscillator.
The method consists in defining the ladder operators subject
to a fundamental quantum request: that the operators
$\hat{\vec{x}}$ and $\hat{\vec{p}}$ (int the dimensionality
of the problem) comply with $[\hat{x}_i, \hat{p}_j] = i
\hbar \delta_{ij}$. This procedure is usually called
canonical quantization and is applied to a great variety of
non-relativistic quantum systems, (almost) all quantum field
theories (or relativistic quantum systems) and even more
complex theories, such as string theory.
Finally, we must mention on an interpreattion of this
canonical quantization in terms of the ladder operators
called the second quantizaton. As we have seen, the energy
of the harmonic oscillator is quantized as
$$
E_n = \hbar\omega(n + 1/2)
$$
which is (almost) identical to Plank's proposed quantization
of the electromagnetic radiation $\hbar\omega n$, except for
the conditional energy shift $\frac{1}{2}\hbar\omega$. In
Einstein's original resolution to the photoelectric effect,
the energy shift does not apper and the value of $n$ counts
the number of electromagnetic-radiation quanta or photons.
Later, Einstein himself realized that an energy shift
proportional to $\frac{1}{2} \hbar \omega$ was needed to
understand e.g., the puzzle of black-body radiation. This
energy shift can be interpreted as the vacuum energy. Thus,
in Heisenberg's formulation, $\vert 0\rangle $ corresponds to the vaccum
state, while the states $\vert 1\rangle $,$\vert 2\rangle $, $\vert 3\rangle $, $\ldots$ may
correspond to a physical system with one, two, three,
$\ldots$ photons or particles, in general. Notice that in
this interpreation $\hat{a}^{\dagger}$ and $\hat{a}$ operate
as creation and annihilation operators
\begin{align}
  \hat{a}^{\dagger} \vert n\rangle  \mapsto \vert n+1\rangle  \nonumber\\
  \hat{a} \vert n\rangle  \mapsto \vert n-1\rangle  \nonumber
\end{align}
creating and annihilating particles. This interpretation
finds an appropiate environment particularly in quantum
field theories, where such processes
% ends pp21
occur naturally. It is important to mention though that, as
long as one considers the non-relativistic Schrödinger wave
equation with a Mermitian potential $\hat{V} =
\hat{V}^{\dagger}$, we shall be interested only in elastic
processes, where the number of particles do not change.
\subsubsection{Coherent states}
We would like to identify non-stationary states of the
harmonic oscillator that behave as much as possible like
classical systems. Schrödinger identified this kind of
states in 1926, searching for solutions of his wave equation
compatible with the correspondence principle (i.e., that in
a limit behave as classical systems). One condition on these
states is then that they do not lose their structure as they
evolve in time, i.e., that their dispersion keeps a minimal
value. A clear example of such a state is the ground state
$\vert 0\rangle $, which in the position space is given by the Gaussian
profile (\ref{equ:2.91})
$$
 \psi_0 (\tilde{x}) = A_0 \exp(-\frac{1}{2}\tilde{x}^2)
$$
Note that, according to (\ref{equ:2.102}), this state has
minimal uncertainty
$$
\langle (\Delta \hat{x}^2)\rangle \langle (\Delta \hat{p})^2\rangle  = \frac{\hbar^2}{4}
$$
For these reasons, coherent states are frequently called
Gaussian states, minimal uncertainty states, and also
canonical coherent states, as there are many other types of
coherent states.
It is possible to show that these canonical coherent states
can be defined as the eigenstates of the annihilation
operator $\hat{a}$
\begin{equation}
  \hat{a}\vert \alpha\rangle  = \alpha\vert \alpha\rangle , \alpha \in \mathbb{C}
  \label{equ:2.106}
\end{equation}
The explicit expression of $\vert \alpha\rangle $ can be found from
comparing

\begin{subequations}
  \begin{alignat}{4}
    \langle n\vert a\vert \alpha\rangle  &= (\langle n\vert a)\vert \alpha\rangle  &&=
    \sqrt{n+1}\langle n+1\vert \alpha\rangle , \label{equ:2.107a}\\
    \text{and} \langle n\vert a\vert \alpha\rangle  &= \langle n\vert (a\vert \alpha\rangle ) &&= \alpha
    \langle n\vert \alpha\rangle , \label{equ:2.107b}
  \end{alignat}
\end{subequations}
where (\ref{equ:2.82b}) has been used. Recalling the form of
$\vert \alpha\rangle $ (\ref{equ:2.105}), we then find that
\begin{equation}
  c_{n+1} = \frac{\alpha}{\sqrt{n+1}}c_n
  \label{equ:2.108}
\end{equation}
where, furthermore, from (\ref{equ:2.80}),
\begin{equation}
  c_n \equiv \langle n\vert \alpha\rangle  =
  \langle 0\vert \frac{(\hat{a})^n}{\sqrt{n!}}\vert \alpha\rangle  =
  \frac{\alpha^n}{\sqrt{n!}} \langle 0\vert \alpha\rangle  =
  \frac{\alpha^n}{\sqrt{n!}}c_0
  \label{equ:2.109}
\end{equation}
It then follows that
\begin{equation}
  \vert \alpha\rangle  = \sum_n c_n \vert n\rangle  = \sum_n
  \frac{\alpha^n}{\sqrt{n!}}c_0
  \frac{(\hat{a}^{\dagger})^n}{\sqrt{n!}} \vert 0\rangle  = c_0
  \exp(\alpha \hat{a}^{\dagger}) \vert 0\rangle 
  \label{equ:2.110}
\end{equation}
(Notice that (\ref{equ:2.108}) is just a consistency check
of (\ref{equ:2.109}).) The constant $c_0$ can be found by
normalizing the state as follows
  \begin{alignat}{2}
    1 = \langle \alpha\vert \alpha\rangle  &= \vert c_0\vert ^2
    \langle 0\vert \exp(\alpha^*\hat{a})\exp(\alpha\hat{a}^{\dagger})\vert 0\rangle 
    \nonumber\\
    &= \vert c_0\vert ^2 \langle 0\vert 
    \Bigl(
      1+\alpha^*\hat{a} + \frac{\alpha*\hat{a}^2}{2} +
      \ldots
    \Bigr)
    \Bigl(
    1+\alpha^*\hat{a}^{\dagger} + \frac{\alpha*(\hat{a}^{\dagger})^2}{2} +
      \ldots
    \Bigr)\vert 0\rangle \nonumber\\
    &=
    \vert c_0\vert ^2\langle 0\vert 
    \Bigl(
      1+\vert \alpha\vert ^2\hat{a}\hat{a}^{\dagger} +
      \frac{(\vert \alpha\vert ^2)^2}{2}\hat{a}^2(\hat{a}^{\dagger})^2
      + \ldots
    \Bigr)\vert 0\rangle \nonumber\\
    &=\vert c_0\vert ^2
    \Bigl(
    1+\vert \alpha\vert ^2 + \frac{1}{2}(\vert \alpha\vert ^2)^2 + \ldots
    \Bigr) = \vert c_0\vert ^2 \exp(\vert \alpha\vert ^2) \label{equ:2.111}
  \end{alignat}
where in the third row we used that
$\langle 0\vert \hat{a}^n(\hat{a}^{\dagger})^m\vert 0\rangle  = 0$ for $n \neq m$, and
in the fourth row that $(\hat{a}\hat{a}^{\dagger})^n = (1+
\hat{a}^{\dagger}\hat{a})^n$ and
$\langle 0\vert \hat{a}^{\dagger}\hat{a}\vert 0\rangle  = 0$. Thus,
eq.~(\ref{equ:110})  becomes
\begin{equation}
  \vert \alpha\rangle  = \exp(-\vert \alpha\vert ^2/2)
  \exp(\alpha\hat{a}^{\dagger})\vert 0\rangle 
  \label{equ:2.112}
\end{equation}
We may also viwe the coherent states as the ground state of
a new set of annihilation and creation operators defined as
\begin{equation}
  \hat{b} = \hat{a} - \alpha \quad \text{and}\quad
  \hat{b}^{\dagger} = \hat{a}^{\dagger} - \alpha^*
  \label{equ:2.113}
\end{equation}
which implies
\begin{equation}
  \hat{b}\vert \alpha\rangle  = 0
  \label{equ:2.114}
\end{equation}

We further observe that in configuration space, the coherent
states are just given by a shifted harmonic-oscillator
ground state
\begin{equation}
  \psi_{\alpha}(x) =
  \Bigl(\frac{m\omega}{\pi\hbar}\Bigr)^{1/4}
  \exp\Bigl(-\frac{m\omega}{2\hbar}(x-x_0)^2\Bigr)
  \label{equ:2.115}
\end{equation}
because
  \begin{alignat}{2}
    \langle x\vert \hat{a}\vert \alpha\rangle  &= \int dx' \, \langle x\vert 
    \sqrt{\frac{m\omega}{2\hbar}} \Bigl(
    \hat{x} + \frac{i\hat{p}}{m\omega}
    \Bigr)\vert x'\rangle  \langle x'\vert \alpha\rangle \nonumber\\
    &=
    \sqrt{\frac{m\omega}{2\hbar}}
    \Bigl(
      x + \frac{\hbar}{m\omega}\frac{d}{dx}
    \Bigr)
    \psi_{alpha}(x)\nonumber\\
    &=
    \sqrt{\frac{m\omega}{2\hbar}}
    \Bigl(
      x-(x-x_0)
      \Bigr)\psi_{\alpha}(x)\nonumber\\
      &=
      x_0
      \sqrt{\frac{m\omega}{2\hbar}} \psi_{\alpha}(x)
      \label{equ:2.116}
  \end{alignat}
while, on the left-hand side we also have
\begin{equation}
  \langle x\vert \hat{a}\vert \alpha\rangle  = \alpha\langle x\vert \alpha\rangle  =
  \alpha\psi_{\alpha}(x)
  \label{equ:2.117}
\end{equation}
yielding
\begin{equation}
  x_0 = \alpha \sqrt{\frac{2\hbar}{m\omega}}
  \label{equ:2.118}
\end{equation}
It is left as an exercise tot he curious reader the explicit
computation of $\langle x\vert \alpha\rangle $, which may be quite illustrative
of the kind of computations one frequently finds in
Heisenberg's formalism.
Notice now that, as expected, the average value of $\hat{x}$
in $\alpha\rangle $ is shifted:
  \begin{alignat}{2}
    \langle \alpha\vert \hat{x}\vert \alpha\rangle  &= 
    \sqrt{\frac{\hbar}{2m\omega}}
    \langle \alpha\vert \hat{a}+\hat{a}^{\dagger}\vert \alpha\rangle  \nonumber\\
    &=
    \sqrt{\frac{\hbar}{2m\omega}}
    (\alpha + \alpha^*) = 
    \sqrt{\frac{\hbar}{2m\omega}} 2 \Re \alpha \nonumber\\
    &=
    \sqrt{\frac{\hbar}{2m\omega}}
    2x_0 \sqrt{\frac{m\omega}{2\hbar}} = x_0,
    \label{equ:2.119}
  \end{alignat}
where we have used (\ref{equ:2.118})
% pp 26
\subsubsection{Harmonic oscillator in configuration space}
In the configuration space, the stationary Schrödinger wave
eq. is given by (\ref{equ:2.42}):
$$
E\psi(x) = - \frac{\hbar^2}{2m} \frac{d^2 \psi}{dx^2} +
\frac{1}{2}m\omega^2 x^2 \psi
$$
Choosing, as before, the dimensionless coordinate
\begin{equation}
  \tilde{x} = \sqrt{\frac{m\omega}{\hbar}} x,
  \label{equ:2.120}
\end{equation}
one obtains
$$
-\frac{\hbar^2}{2m} \frac{m\omega}{\hbar} \frac{d^2 \psi}{d
\tilde{x}^2} + \frac{1}{2} m \omega^2 \frac{\hbar}{m\omega}
\tilde{x}^2 \psi = E\psi
$$
which implifies to
\begin{equation}
  \psi'' (\tilde{x}) - \tilde{x}^2 \psi(\tilde{x}) = -
  \frac{2E}{\hbar\omega} \psi(\tilde{x})
  \label{equ:2.121}
\end{equation}
Motivated by the fact that the homogeneous analog of
(\ref{equ:2.121}) has the solution $\psi_{\text{homog}} =
\exp\Bigl(-\frac{1}{2}\tilde{x}^2\Bigr)$, we propose the
ansatz
\begin{equation}
  \psi(\tilde{x}) = A \exp(-\frac{1}{2}\tilde{x}^2)
  u(\tilde{x})
  \label{equ:122}
\end{equation}
Inserting (\ref{equ:2.122}) in (\ref{equ:2.121}), we find
\begin{equation}
  A \exp\Bigr(-\frac{1}{2}\tilde{x}^2\Bigl) \Bigl(
  u'' - 2\tilde{x}u'-u+\tilde{x}u-\tilde{x}u +
  \frac{2E}{\hbar\omega}u
  \Bigr) = 0,
  \label{equ:2.123}
\end{equation}
wich can be rewritten simply as
\begin{equation}
  u'' - 2\tilde{x}u' + (\lambda -1)u = 0 \quad \text{with}
  \quad \lambda = \frac{2E}{\hbar\omega}
  \label{equ:2.124}
\end{equation}
A very similar differential equation was found by Hermite:
\begin{equation}
  u'' - 2\tilde{x}u' + 2nu = 0,
  \label{equ:2.125}
\end{equation}
who also demonstrated that $n \in \mathbb{Z}$ for
(\ref{equ:2.125}) to have solutions\footnote{See \S 13 of
Arfken, Math. Meth. for Physicists. 1985; app A2 of De la
Peña, Introducción a la Mecánica Cuántica}.
This implies that
\begin{equation}
  \lambda_n - 1 = 2n \Leftrightarrow
  \frac{2E_n}{\hbar\omega} = 2n+1,\quad n \in \mathbb{Z}
  \label{equ:2.125}
\end{equation}
which coincides with the result we obtained before in terms
of the number operator. The solution of (\ref{equ:2.125})
correspond to the Hermite polynomials, given by
(\ref{equ:2.97}) or equivalently by
\begin{equation}
  u(\tilde{x}) = H_n(\tilde{x}) = (-1)^n \exp(\tilde{x}^2)
  \frac{d^n}{d\tilde{x}^n} \exp(-\tilde{x}^2), \qquad n \in
  \mathbb{Z}
  \label{equ:2.126}
\end{equation}
From our ansatsz (\ref{equ:2.122}), we obtain
\begin{equation}
  \psi_n(\tilde{x}) = A_n
  \exp(-\frac{1}{2}\tilde{x}^2)H_n(\tilde{x})
  \label{equ:2.127}
\end{equation}
The normalization constants $A_n$ can be computed by
applying the orthogonality relation of Hermite polynomials
\begin{equation}
  \int_{-\infty}^{\infty} d\tilde{x} \,
  \exp(-\tilde{x}^2) H_n(\tilde{x}) H_m(\tilde{x}) = 2^n n!
  \sqrt{\pi} \delta_{n,m},
  \label{equ:2.128}
\end{equation}
as follows
$$
1 = \int_{-\infty}^{\infty} \psi_n^*(x) \psi_{n'} (x) =
\sqrt{\frac{\hbar}{m\omega}} \int_{-\infty}^{\infty}
d\tilde{x} \, \vert A_n\vert ^2 \exp(-\tilde{x}^2) H_n^2(\tilde{x}) =
\vert A_n\vert ^2 \sqrt{\frac{\hbar}{m\omega}} 2^n n! \sqrt{n}
$$
Thus, we have
\begin{equation}
  A_n = \Bigl(\sqrt{\frac{\hbar \pi}{m\omega}}2^n n!\Bigr)^{-1/2}
  \label{equ:2.129}
\end{equation}
up to an unphysical phase\footnote{In QM, a physical state
is not represented by a specific normalized vector on Hilber
space, but by a ray, i.e., a class of all vectors differeing
only a phase factors. So, these phases are not measurable
and thus unphysical.}. Note that eqs. (\ref{equ:2.129}) and
(\ref{equ:2.127}) yield our previous result,
(\ref{equ:2.96}).
Besided the contact of this formalism with some common
mathematical tool, it is hard to identify a reason to prefer
this method before Heisenberg's method with Dirac's ladder
operators.
%
\subsubsection{Free particle in one dimension}
In contrast to the harmonic oscillator, a free particle has
no potential bo bind it. Thus, it cannot be confined and can
therefore have arbitrary energy. This is the simplest
quantum system, but it is useful to make a couple of
remarks.
The Schrödinger's wave eq. is
\begin{equation}
  \frac{d^2 \psi}{d x^2} + \frac{2m}{\hbar^2} E \psi = 0
  \label{equ:2.130}
\end{equation}
We can define the wave number
\begin{equation}
  k^2 = \frac{2mE}{\hbar^2} \rangle  0,
  \label{equ:1.131}
\end{equation}
which simplifies (\ref{equ:2.130}) to
\begin{equation}
  \frac{d^2 \psi}{dx^2 } + k^2 \psi = 0,
  \label{equ:1.132}
\end{equation}
whose solution is
\begin{equation}
  \psi(x) = A \exp(ikx) + B \exp(-ikx)
  \label{equ:2.133}
\end{equation}
From (\ref{equ:2.41}), we can easily obtain the time
dependence of the wave-function as
\begin{equation}
  \psi(x,t) = \exp(-i\omega t) \psi(x) = A
  \exp\Bigl(i(kx-\omega t)\Bigr)
  +
  B
  \exp\Bigl(-i(kx+\omega t)\Bigr)
  \label{equ:2.134}
\end{equation}
where we defined $\omega = E/\hbar$ in agreement with
Plack's hypothesis.
We realize that the first term is right-moving ``plane"-wave
with phase velocity $\omega / k$ whereas the second term
corresponds to a left-moving ``plane"-wave. Clearly, left
and right waves can be treated independently, reason why
many authors disregard one of both waves.
Let us now compute the momentum associated with $\psi(x)$ in
(\ref{equ:2.133}). In the configuration space, we have
\begin{equation}
  -i\hbar \frac{d}{dx} \psi(x) = -i\hbar(ik) \Bigl(A
  \exp(ikx) - B\exp(-ikx)\Bigr)
  \label{equ:2.135}
\end{equation}
where, as expected, we see that the momentum of the  first
term is opposite to the one of the second term, and we
identify the relation
\begin{equation}
  p = \pm \hbar k \pm \sqrt{2mE}
  \label{equ:2.136}
\end{equation}
where the sign depends on the motion of the wave.
Clearly, the normalization of the solution (\ref{equ:2.133})
cannot be done in a traditional from by simply setting
$$
\int_{-\infty}^{\infty} \psi^*(x) \psi(x)\, dx = 1
$$
Dirac suggested to normalize by using a sort of
orthonormality realtion. For simplicity, let us consider
only a right-moving wave rewritten as
\begin{equation}
  \psi_p(x) = A \exp\Bigl(\frac{i}{\hbar}px\Bigr)
  \label{equ:2.137}
\end{equation}
where the index $p$ labels the solution with momentum $p$.
Let us compute
\begin{equation}
  \int_{-\infty}^{\infty} \psi_p^*(x) \psi_{p'}(x) \, dx =
  \vert A\vert ^2 \int_{-\infty}^{\infty}
  \exp\Bigl(\frac{i}{\hbar}x(p'-p)\Bigr) \, dx = \vert A\vert ^2
  (2\pi\hbar)\delta(p'-p)
  \label{equ:2.138}
\end{equation}
If we wish this eq. to be an orthonormality relation, the
result should be $\delta(p'-p)$. Therefore, we identify
\begin{equation}
  A = \frac{1}{\sqrt{2\pi\hbar}}
  \label{equ:2.139}
\end{equation}
up to an unphysical phase. A similar discussion can be done
for the left-moving part of $\psi(x)$. We obtain then
\begin{equation}
  \psi(x,t) = \frac{1}{\sqrt{2\pi\hbar}}
  \Bigl[
    \exp\bigl(i(kx-\omega t)\bigr) +
    \exp\bigl(-i(kx+\omega t)\bigr)
  \Bigr]
  \label{equ:2.140}
\end{equation}
It is interesting to solve the problem in the momentum
space. Recalling that (or simply using (\ref{equ:2.59}))
\begin{equation}
  \langle p\vert \frac{\hat{p}^2}{2m}\vert \alpha\rangle  =
  \frac{p^2}{2m} \langle p\vert \alpha\rangle  = \frac{p^2}{2m} \phi_{\alpha}(p)
  \label{equ:2.141}
\end{equation}
we find that the Schrödinger's wave equation to consider
reads
\begin{equation}
  \Bigl(\frac{p^2}{2m}-E\Bigr)
  \phi(p) = 0
  \label{equ:2.142}
\end{equation}
It is clear then that $\phi(p) = 0$ for all $p$, excepting
when $p = \pm \sqrt{2mE} = \pm \hbar k$, coinciding with
(\ref{equ:2.136}). That is,
\begin{equation}
  \phi(p) = a \delta(p - \hbar k) + b \delta (p+k\hbar)
  \label{equ:2.143}
\end{equation}
Notice that this result can equally be obtained by applying
(\ref{equ:2.57b}) to (\ref{equ:2.133}) with the normalizatio
constant (\ref{equ:2.139})
  \begin{alignat}{2}
    \phi(p) &= \int dx \frac{1}{\sqrt{2\pi\hbar}}
    \exp(-ipx/\hbar) \frac{1}{\sqrt{2\pi\hbar}}
    \Bigl[\exp(ikx) + \exp(-ikx)\Bigr] \\
    &=
    \frac{1}{2\pi\hbar} \int dx \,
    \biggl[
      \exp\Bigl(-\frac{i}{\hbar}(p-k\hbar)x\Bigr)+
      \exp\Bigl(-\frac{i}{\hbar}(p+k\hbar)x\Bigr)
    \biggr]\\
    &=
    \delta(p - k\hbar) + \delta(p + k\hbar)
  \end{alignat}
and thus in the momentum representation, the wave function
is
\begin{equation}
  \phi(p,t) = \exp(-i\omega t) \Bigl[\delta(p - k\hbar) +
  \delta(p+k\hbar)\Bigr]
  \label{equ:2.145}
\end{equation}
A final remark on the one-dimensional free particle concerns
the equivalence of different potentials. It is evident that
in the configuration space a constant shift in the energy by
$V(x) = V_0 \in \mathbb{R}$ yields the same physics.
According to (\ref{equ:2.60}), in the momentum space,
although $V(x) = 0$ corresponds to $V(p) = 0$, $V(x) = V_0$
corresponds to
\begin{equation}
  V(p-p') = \frac{1}{\sqrt{2\phi\hbar}}
  \int dx\,
  \exp\bigl(-i(p-p')x/\hbar\bigr) V_0 = V_0 \delta(p-p')
  \label{equ:2.146}
\end{equation}
which yields the wave equation
$$
\frac{p^2}{2m} \phi(p) + \int dp' \,
\frac{1}{\sqrt{2\pi\hbar}} V_0 \delta(p-p') \phi(p') =
E\phi(p),
$$
that simplifies to
\begin{equation}
  \frac{p^2}{2m} \phi(p) + V_0 \phi(p) = E\phi(p)
  \label{equ:2.147}
\end{equation}
Remark that (\ref{equ:2.147}) is very close to the analogous
equation in the configuration space:
\begin{equation}
  -\frac{\hbar^2}{2m} \frac{d^2}{dx^2} \psi(x) + V_0 \psi(x)
  = E \psi(x)
  \label{equ:2.148}
\end{equation}
\subsubsection{Linear potential}
Let us study now a potential, whose solution will be useful
whe dealing with approximative methods, such as the WKB
approximation that we shall study latter. The linear
potential is given by
\begin{equation}
  V(x) = k|x
  \label{equ:2.149}
\end{equation}
and is depicted in the figure %todo add figure
This potential is symmetric, i.e., invariant under space
invsersion
$$
x \to -x
$$
which can be written in terms of the unitary parity operator
$\hat{P}$ such that, in terms of operators,
\begin{equation}
  \hat{P}^{-1} \hat{x} \hat{P} = - \hat{x}
  \label{equ:2.150}
\end{equation}
and clearly satisfies
\begin{equation}
  \hat{P}^2 = \mathbb{1} \qquad\Rightarrow\qquad
  \hat{P}^{-1} = \hat{P}
  \label{equ:2.151}
\end{equation}
An operator $\hat{A}$, such that
\begin{equation}
  \hat{P}\hat{A}\hat{P} = -\hat{A}
  \label{equ:2.152}
\end{equation}
is said to be odd under parity. If $\hat{A}$ is left invariant
under the action of $\hat{P}$, then $\hat{A}$ is even. The
position operator is odd.
One can further conceive the eigenstates of $\hat{P}$ as
\begin{equation}
  \hat{P} \vert \pm | \rangle = \pm 1 \vert \pm | \rangle
  \label{equ:2.153}
\end{equation}
where, an analogy with the operators, or state $\vert -1
\rangle$ ($\vert + 1 \rangle$) is said to be odd (even)
under parity.
In the special situation when the Hamiltonian is parity
invariant
\begin{equation}
  \hat{P} \hat{H} \hat{P} = \hat{H} \qquad \Rightarrow \qquad
  [\hat{P}, \hat{H}] = 0,
  \label{equ:2.154}
\end{equation}
that is, the (even or odd) eigenstates of $\hat{P}$ are also
energy eigenstates.
Furthermore, as we shall se in the next session, $[\hat{P},
\hat{P}] = 0$ implies thath parity is conserved. It then
follows that stationary states of an even
parity/parity-invariant Hamiltonian are eiger even or odd.
For the Hamiltonian
$$
\hat{H} = \frac{\hat{p}^2}{2m} + \hat{V}(|\hat{x}|),
$$
we can see that (\ref{equ:2.152}) is satisfied independently
of the parity of $\hat{p}$. However, as we shall also see in
the next section, for a large class of models
\begin{equation}
  \hat{p} = \frac{im}{\hbar}[\hat{H}, \hat{x}]
  \label{equ:2.155}
\end{equation}
from which we observe that
\begin{equation}
  \hat{P} \hat{p} \hat{P} = \frac{im}{\hbar}
  \hat{P}[\hat{H}, \hat{x}] \hat{P} =
  \frac{im}{\hbar} [\hat{H}, \hat{P}\hat{x}\hat{P}] =
  -\hat{p}
  \label{equ:2.156}
\end{equation}
Turning back to our problem, with the potential
(\ref{equ:2.149}), we can focus separately in even and odd
solution $\psi(x)$ of
\begin{equation}
  -\frac{\hbar^2}{2m} \frac{d^2 \psi(x)}{dx^2} + k|x|\psi(x)
  = E \psi(x)
  \label{equ:2.157}
\end{equation}
Besides demanding that $\psi(x \to \infty) = 0$, we impose
for odd solutions that
\begin{subequations}
\begin{alignat}{2}
  \psi(0) &= 0 \qquad \text{because} \quad \psi(0) = -
  \psi(-0)
  \label{equ:2.158a}\\
  \intertext{and for even solutions that}
  \psi'(x = 0) &= 0 \qquad \text{because} \quad \psi'(0) =
  \lim_{\varepsilon \to 0}
  \frac{\psi(\varepsilon)-\psi(-\varepsilon)}{\varepsilon} =
  0
  \label{equ:2.158b}
\end{alignat}
\end{subequations}
for an infinitesimally small $\varepsilon$.
We can rewrite (\ref{equ:2.157}) in terms of the
dimensionless coordinate and energy
\begin{equation}
  \tilde{x} = \Bigl(\frac{\hbar^2}{mk}\Bigr)^{-1/3}x, \quad
  \tilde{E} = \Bigl(\frac{\hbar^2 k^2}{m}\Bigr)^{-1/3} E,
  \label{equ:2.159}
\end{equation}
as
\[
  -\frac{\hbar^2}{2m}
  \Bigl(\frac{\hbar^2}{mk}\Bigr)^{-2/3}
  \frac{d^2 \psi}{d\tilde{x}^2}
  +
  \Bigl(\frac{\hbar^2}{mk}\Bigr)^{1/3}k|\tilde{x}| \psi
  =
  \Bigl(\frac{\hbar^2 k^2}{m}\Bigr)^{1/3}\tilde{E} \psi
\]
%ends 32
