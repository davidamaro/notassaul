\section{Schrödinger equation}
\subsection{Time evolution}
In qm time is a paramter and not and observable/operator.
$H$ labels a state given at a time. This implies that we may
envirage that, if a physical system is represented by
$|\alpha>$ at the time $t_0$, at a later time, the state may
differ. The new "evolved" state from $t_0$ to $t$ can be
denoted by
$$
|\alpha,t_0;t>, \quad t \geq t_0
$$
such that
$$
|\alpha,t_0;t_0> = |\alpha>
$$
The transition from $|\alpha>$ to $|\alpha,t_0;t>$ is given
by the so-called time-evolution operator $\hat{U}(t,t_0)$,
such that
\begin{equation}
  |\alpha,t_0;t> = \hat{U}(t,t_0) |\alpha>
  \label{equ:2.1}
\end{equation}
The time-evolution operator has the following properties:
\begin{itemize}
  \item Unitarity. Suppose that at $t_0$ $|\alpha>$ is
    expanded in terms of the (orthonormal) eigenstates $|a>$
    of some observable $A$
    \begin{equation}
      |\alpha> = \sum_a |a><a|\alpha,t_0;t> = \sum_a
      c_a(t_0)|a>
      \label{equ:2.2}
    \end{equation}
    Thus, at a later time $t > t_0$ we have
    \begin{equation}
      |\alpha,t_0;t> = \sum_a c_a(t) |a>
      \label{equ:2.3}
    \end{equation}
    where $c_a(t)$ is the prob. amp. of finding $|\alpha>$
    at $|a>$.
    Although we expect that in general for any particular
    $a$, the prob. amplitudes differ,
    $$-
      c_a(t) \neq c_a(t_0)
    $$
    the sum of all probabilies must be unity at all times,
    i.e.,
    \begin{equation}
      \sum_a |c_a(t)|^2 = \sum_a |c_a(t_0)|^2 = 1
      \label{equ:2.4}
    \end{equation}
    (as long as the states $|\alpha>$ and $|a>$ are
    "correctly" normalized).
    This implies that
    \begin{equation}
      <\alpha|\alpha> = \sum_{a'} \sum_a c_{a'}^*(t_0)
      <a'|a>  c_a(t_0) = \sum_a |c_a(t_0)|^2 = 1
      \label{equ:2.5}
    \end{equation}
    and consequently
    \begin{equation}
      <\alpha,t_0;t|\alpha,t_0;t> = \sum_a |c_a(t)|^2 = 1
      \label{equ:2.6}
    \end{equation}
    That is, the state $|\alpha>$ remain normalized at all
    times. For terms of \ref{equ:2.1}, this is satisfied as
    long as
    \begin{equation}
      \hat{U^\dagg}(t, t_0) \hat{U}(t,t_0) = 1
      \label{equ:2.7}
    \end{equation}
  \item Time composition:
    \begin{equation}
      \hat{U}(t_2,t_0) = \hat{U}(t_2,t_1) \hat{U}(t_1,t_0)
      \qquad t_0 < t_1 < t_2
      \label{equ:2.8}
    \end{equation}
  \item Infinitesimal form. We can also propose an
    infinitesimal form for $\hat{U}(t,t_0)$, such that
    \begin{equation}
      |\alpha,t_0;t_0 + dt> = \hat{U}(t_0+dt,t_0) |\alpha>,
      \qquad \lim_{dt \to 0} \hat{U}(t_0+dt,t_0) = 1
      \label{equ:2.9}
    \end{equation}
    We can note that this together with the previous
    properties is satisfied if
    \begin{equation}
      \hat{U}(t_0+dt,t_0) = 1- i\hat{\Omega)dt}
      \label{equ:2.10}
    \end{equation}
    when $\hat{\Omega}$ is a Hermitian operator.
    This can be shown as follows. Notice that
    \begin{align*}
      \hat{U^\dagg} (t_0+dt,t_0) \hat{U}(t_0+dt,t_0) &= 
      (1+ i\hat{\Omega)dt})(1- i\hat{\Omega)dt})\\
      &=
      1+i(\hat{\Omega^\dagg}-\hat{\Omega})dt + O(dt^2)
    \end{align*}
    which equals the identity as long as $\hat{\Omega} =
    \hat{\Omega^\dagg}$ (Hermitian) and $dt^2$ is
    negligilble. Furthermore, the time composition is also
    direct
    \begin{align*}
      \hat{U}(t_0+dt_1+dt_2,t_0+dt_1)\hat{U}(t_0+dt_1,t_0)
      &= (1-i\hat{\Omega}dt_2)(1-i\hat{\Omega}dt_1) \\
      &= 1-i\hat{\Omega}(dt_2+dt_1) + O(dt_1dt_2)\\
      &= \hat{U}(t_0+dt_1+dt_2,t_0)
    \end{align*}
    Borrwing from classical mechanics the idea that the
    Hamiltonian $H$ is the generator of time evolution, we
    can asser that
    \begin{equation}
      \hat{\Omega} = \frac{\hat{H}}{\hbar}
      \label{equ:2.11}
    \end{equation}
    which has the righ dimensions of frequency. Therefore,
    we obtain
    \begin{equation}
      \hat{U}(t_0+dt,t_0) = 1 -\frac{i}{\hbar}\hat{H}dt
      \label{equ:2.12}
    \end{equation}
    for (\ref{equ:2.11}) we have used $\hbar$ to get the
    right dimensions, recalling the Planck radiation $E =
    \hbar \omega$.
\end{itemize}
These properties lead to and interesting differential
quation. Using the time decomposition property, we see that
\begin{equation}
  \hat{U}(t+dt,t_0) = \hat{U}(t+dt,t) \hat{U}(t,t_0) =
  (1-\frac{i}{\hbar}\hat{H}dt)\hat{U}(t,t_0)
  \label{equ:2.13}
\end{equation}
Note now that
\begin{equation}
  \hat{U}(t+dt,t_0)-\hat{U}(t,t_0) = - \frac{i}{\hbar}
  \hat{H}dt \hat{U}(t,t_0)
  \label{equ:2.14}
\end{equation}
which implies
$$
i\hbar \frac{\hat{U}(t+dt,t_0)-\hat{U}(t,t_0)}{dt} =
\hat{H}\hat{U}(t,t_0)
$$
that can be written in the differential form
\begin{equation}
  i\hbar \frac{\partial}{\partial t}\hat{U}(t_0,t) = \hat{H}
  \hat{U}(t,t_0)
  \label{equ:2.15}
\end{equation}
This is the Schrödinger equation for the time-evolution
operator and is the most fundamental equation of quantum
mechanics.
Multiplying (\ref{equ:2.15}) by the ket $|\alpha> =
%todo no entiendo que es lo que dice el antes del we find
|\alpha,t_0;t_1>$, we find
$$
i\hbar \frac{\partia}{\partial t} \hat{U}(t,t,0) |\alpha> =
\hat{H}\hat{U}(t,t_0) |\alpha>
$$
\begin{equation}
  i\hbar \frac{\partial}{\partial t} |\alpha,t_0;t> =
  \hat{H}|\alpha,t_0;t>
  \label{equ:2.16}
\end{equation}
which is the Schrödinger eq. for a state.

The Hamiltonian $\hat{H}$ has in principle an arbitrary form
(it can be an abstract op., or a differential op., or have
matrix form) and be time-dependent or time-independent. We
can straightforwardly obtain the form of $\hat{U}(t,t_0)$
for a time-independent Hamiltonian by solving (\ref{2.15}):
\begin{equation}
  \hat{U}(t,t_0) = \exp(-\frac{i}{\hbar}H(t-t_0))
  \label{equ:2.17}
\end{equation}
which makes sense in general as the series
\begin{equation}
  \hat{U}(t,t_0) = 1 -
  \frac{i}{\hbar}\hat{H}(t-t_0)+\frac{1}{2!}\Big(\frac{-i}{\hbar}\hat{H}(t-t_0)\Big)^2
  \label{equ:2.18}
\end{equation}
A time-dependent Hamiltonian is more complicated. We must
distinguish two cases:
\begin{itemize}
  \item when $[\hat{H}(t_1), \hat{H}(t_2)] = 0$, and
  \item when $[\hat{H}(t_1), \hat{H}(t_2)] \neq 0$
\end{itemize}
In the former case, the solution of (\ref{equ:2.15}) reads
\begin{equation}
  \hat{U}(t,t_0) = \exp\Big(-\frac{i}{\hbar}\int_{t_0}^t dt'
  \, \hat{H}(t')\Big)
  \label{equ:2.19}
\end{equation}
%% todo a partir de aquí saúl volvió a escrbir 2.18
To solve (\ref{equ:2.15}) when $[\hat{H}(t_1), \hat{H}(t_2)]
\neq 0$, we first point out that this differential equation
can be restated as
$$
\hat{U}(t,t_0) \bigg|_{t=t_0}^t = 
-\frac{i}{\hbar} \int_{t_0}^t dt' \,
\hat{H}(t')\hat{U}(t',t_0)
$$
recalling that $\hat{U}(t_0,t_0) = 1$ we then have
\begin{equation}
  \hat{U}(t,t_0) = 1 - \frac{i}{\hbar} \int_{t_0}^t dt' \,
  \hat{H}(t')\hat{U}(t',t_0)
  \label{equ:2.20}
\end{equation}
By iteration, we can find the approximate solution
% todo arreglar este desmadre con la numeración
% todo me dí cuenta de que estoy usando mal los bigg
\begin{align}
  \hat{U}(t,t_0) &= 1 - \frac{i}{\hbar} \int_{t_0}^t dt'\,
  \hat{H}(t')\bbigr[1 - \frac{i}{\hbar}
  \int_{t_0}^{t'} dt'\, \hat{H}(t'')\hat{U}(t'',t_0)\biggl]
  \\
  &=
  1 - \frac{i}{\hbar} \int_{t_0}^t dt' \, \hat{H}(t') + 
  \biggl(\frac{-i}{\hbar}\biggr)^2 \int_{t_0}^t dt' \,
  \int_{t_0}^{t'} dt'' \, \hat{H}(t') \hat{H}(t'')
  \hat{U}(t'',t_0)\\
  &=
  1 + \sum_{n=1}^{\infty} 
  \biggl(\frac{-i}{\hbar}\biggr)^n
  \int_{t_0}^t dt_1\,
  \int_{t_0}^{t_1} dt_2\,
  \cdots
  \int_{t_0}^{n-1} dt_n\,
  \hat{H}(t_1) \hat{H}(t_2) \cdots \hat{H}(t_n)
  %% todo poner 2.21
\end{align}
well-known as the Dyson series.
\subsection{Time evolution of stationary energy eigenstates}
Let us suppose that the Hamiltonian is time-independent. We
would like to figure out the evolution of an arbitrary state
$|\alpha>$. This is particularly simple if we expand
$|\alpha>$ in the basis of eigenstates $|a>$ of an operator
$\hat{A}$, such that
\begin{equation}
  [\hat{A}, \hat{H}] = 0
  \label{equ:2.22}
\end{equation}
i.e., compatible with $\hat{H}$. It follows that $\{|a>\}$
are energy eigenstates too, whose energy eigenvalues are
\begin{equation}
  \hat{H} |a> = E_a |a>
  \label{equ:2.23}
\end{equation}

Expanding the time-evolution operator in the states $|a>$,
we find
%% todo areglar el alig para la numeración
\begin{align*}
  \hat{U}(t, 0) = \exp\Big(-\frac{i}{\hbar} \hat{H}t\Big) &= 
  \sum_{a'} |a'><a'| \exp(-i \hat{H} t / \hbar) \sum_a
  |a><a|\\
  &=
  \sum_{a,a'} |a'><a| <a'|\exp(-i\hat{H}t/\hbar) |a>\\
  % todo insertar braces
  &=
  \sum_a \exp(-i E_a t /\hbar) |a><a|
  %% todo numeración 2.23
\end{align*}
Now expanding $|\alpha>$ in the basis of energy eigenstates,
we see that
\begin{align*}
|\alpha> &= \sum_a <a|\alpha> |a> = \sum_a c_a(t=0) |a> \\
  \Rightarrow |\alpha;t> &= U(t,0) |\alpha> = \sum_{a'}
  \exp\Big(-i E_{a'}t/\hbar\Big) |a'><a'|a> c_a(t=0) \\
  &=
  \sum_a c_a(t= 0) \exp(-iE_a t/\hbar) |a>\\
  % todo poner símbolo definición
  = \sum_a c_a(t) |a>
  % 2.24
\end{align*}
where we realize the time evolution of the expansion
coefficients.

Notice that for $|\alpha> = |a>$, we find that
\begin{equation}
  |a;t> = \exp(-i E_a t/\hbar) |a>
  \label{equ:2.25}
\end{equation}
so that any simultaneous eigenstate of $H$ and $A$ remains
so at all time.
\subsection{Expectation values}
We are interested in the expectation value of an observable
$B$ in the state
$$
  |a;t> = \hat{U}(t,0) |a>
$$
as given in (\ref{equ:2.26}). The expectation value is
computed as
$$
<\hat{B}> = <a|\hat{U}^{\dagg}(t,0) \hat{B} \hat{U}(t,0) |a>
= 
<a|\exp(iE_a t/\hbar) \hat{B} \exp(-i E_a t/\hbar) |a>
$$
