\section{Schrödinger equation}
\subsection{Time evolution}
In qm time is a paramter and not and observable/operator.
$H$ labels a state given at a time. This implies that we may
envirage that, if a physical system is represented by
$|\alpha>$ at the time $t_0$, at a later time, the state may
differ. The new "evolved" state from $t_0$ to $t$ can be
denoted by
$$
|\alpha,t_0;t>, \quad t \geq t_0
$$
such that
$$
|\alpha,t_0;t_0> = |\alpha>
$$
The transition from $|\alpha>$ to $|\alpha,t_0;t>$ is given
by the so-called time-evolution operator $\hat{U}(t,t_0)$,
such that
\begin{equation}
  |\alpha,t_0;t> = \hat{U}(t,t_0) |\alpha>
  \label{equ:2.1}
\end{equation}
The time-evolution operator has the following properties:
\begin{itemize}
  \item Unitarity. Suppose that at $t_0$ $|\alpha>$ is
    expanded in terms of the (orthonormal) eigenstates $|a>$
    of some observable $A$
    \begin{equation}
      |\alpha> = \sum_a |a><a|\alpha,t_0;t> = \sum_a
      c_a(t_0)|a>
      \label{equ:2.2}
    \end{equation}
    Thus, at a later time $t > t_0$ we have
    \begin{equation}
      |\alpha,t_0;t> = \sum_a c_a(t) |a>
      \label{equ:2.3}
    \end{equation}
    where $c_a(t)$ is the prob. amp. of finding $|\alpha>$
    at $|a>$.
    Although we expect that in general for any particular
    $a$, the prob. amplitudes differ,
    $$-
      c_a(t) \neq c_a(t_0)
    $$
    the sum of all probabilies must be unity at all times,
    i.e.,
    \begin{equation}
      \sum_a |c_a(t)|^2 = \sum_a |c_a(t_0)|^2 = 1
      \label{equ:2.4}
    \end{equation}
    (as long as the states $|\alpha>$ and $|a>$ are
    "correctly" normalized).
    This implies that
    \begin{equation}
      <\alpha|\alpha> = \sum_{a'} \sum_a c_{a'}^*(t_0)
      <a'|a>  c_a(t_0) = \sum_a |c_a(t_0)|^2 = 1
      \label{equ:2.5}
    \end{equation}
    and consequently
    \begin{equation}
      <\alpha,t_0;t|\alpha,t_0;t> = \sum_a |c_a(t)|^2 = 1
      \label{equ:2.6}
    \end{equation}
    That is, the state $|\alpha>$ remain normalized at all
    times. For terms of \ref{equ:2.1}, this is satisfied as
    long as
    \begin{equation}
      \hat{U^\dagg}(t, t_0) \hat{U}(t,t_0) = 1
      \label{equ:2.7}
    \end{equation}
  \item Time composition:
    \begin{equation}
      \hat{U}(t_2,t_0) = \hat{U}(t_2,t_1) \hat{U}(t_1,t_0)
      \qquad t_0 < t_1 < t_2
      \label{equ:2.8}
    \end{equation}
  \item Infinitesimal form. We can also propose an
    infinitesimal form for $\hat{U}(t,t_0)$, such that
    \begin{equation}
      |\alpha,t_0;t_0 + dt> = \hat{U}(t_0+dt,t_0) |\alpha>,
      \qquad \lim_{dt \to 0} \hat{U}(t_0+dt,t_0) = 1
      \label{equ:2.9}
    \end{equation}
    We can note that this together with the previous
    properties is satisfied if
    \begin{equation}
      \hat{U}(t_0+dt,t_0) = 1- i\hat{\Omega)dt}
      \label{equ:2.10}
    \end{equation}
    when $\hat{\Omega}$ is a Hermitian operator.
    This can be shown as follows. Notice that
    \begin{align*}
      \hat{U^\dagg} (t_0+dt,t_0) \hat{U}(t_0+dt,t_0) &= 
      (1+ i\hat{\Omega)dt})(1- i\hat{\Omega)dt})\\
      &=
      1+i(\hat{\Omega^\dagg}-\hat{\Omega})dt + O(dt^2)
    \end{align*}
    which equals the identity as long as $\hat{\Omega} =
    \hat{\Omega^\dagg}$ (Hermitian) and $dt^2$ is
    negligilble. Furthermore, the time composition is also
    direct
    \begin{align*}
      \hat{U}(t_0+dt_1+dt_2,t_0+dt_1)\hat{U}(t_0+dt_1,t_0)
      &= (1-i\hat{\Omega}dt_2)(1-i\hat{\Omega}dt_1) \\
      &= 1-i\hat{\Omega}(dt_2+dt_1) + O(dt_1dt_2)\\
      &= \hat{U}(t_0+dt_1+dt_2,t_0)
    \end{align*}
    Borrwing from classical mechanics the idea that the
    Hamiltonian $H$ is the generator of time evolution, we
    can asser that
    \begin{equation}
      \hat{\Omega} = \frac{\hat{H}}{\hbar}
      \label{equ:2.11}
    \end{equation}
    which has the righ dimensions of frequency. Therefore,
    we obtain
    \begin{equation}
      \hat{U}(t_0+dt,t_0) = 1 -\frac{i}{\hbar}\hat{H}dt
      \label{equ:2.12}
    \end{equation}
    for (\ref{equ:2.11}) we have used $\hbar$ to get the
    right dimensions, recalling the Planck radiation $E =
    \hbar \omega$.
\end{itemize}
These properties lead to and interesting differential
quation. Using the time decomposition property, we see that
\begin{equation}
  \hat{U}(t+dt,t_0) = \hat{U}(t+dt,t) \hat{U}(t,t_0) =
  (1-\frac{i}{\hbar}\hat{H}dt)\hat{U}(t,t_0)
  \label{equ:2.13}
\end{equation}
Note now that
\begin{equation}
  \hat{U}(t+dt,t_0)-\hat{U}(t,t_0) = - \frac{i}{\hbar}
  \hat{H}dt \hat{U}(t,t_0)
  \label{equ:2.14}
\end{equation}
which implies
$$
i\hbar \frac{\hat{U}(t+dt,t_0)-\hat{U}(t,t_0)}{dt} =
\hat{H}\hat{U}(t,t_0)
$$
